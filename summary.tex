% https://www.nature.com/nature/for-authors/formatting-guide

% Articles start with a fully referenced summary paragraph, ideally of no more
% than 200 words, which is separate from the main text and avoids numbers,
% abbreviations, acronyms or measurements unless essential. It is aimed at
% readers outside the discipline.

% This summary paragraph should be structured as follows:
% - 2-3 sentences of basic-level introduction to the field;
% - a brief account of the background and rationale of the work;
% - a statement of the main conclusions (introduced by the phrase 'Here we show' or its equivalent);
% - 2-3 sentences putting the main findings into general context so it is
%   clear how the results described in the paper have moved the field forwards.

% Please refer to our annotated example  to see how the summary paragraph should
% be constructed. https://www.nature.com/documents/nature-summary-paragraph.pdf


% - 2-3 sentences of basic-level introduction to the field;
Array computation libraries provide powerful notation for accessing,
manipulating, and operating on data in vectors, matrices, and
higher-dimensional arrays \cite{iverson1980notation}.
% - a brief account of the background and rationale of the work;
NumPy is the fundamental array computation library for the Python ecosystem
\cite{dubois2007guest,oliphant2007python,millman2011python,perez2011python}.
It has been actively developed for over two decades and has more than 850
contributors, nearly 300,000 dependent repositories on GitHub, and millions of
downloads per year.
It plays an essential role in research analysis pipelines in fields as
diverse as physics, biology, astronomy, neuroscience, material science,
engineering, and chemistry.
For example, in astronomy, NumPy was fundamental to discovering gravitational
waves\cite{pycbc}, the first imaging of a black hole\cite{eht-imaging}, and
continues underpinning the future exploration of our
universe\cite{jenness2018lsst}.
It is one of the most commonly used machine learning tools among
enterprises \cite{451report2018}.
% - a statement of the main conclusions (introduced by the phrase 'Here we show' or its equivalent);
Here we show how a small group of students and scientists created this
popular tool, explain the core ideas that helped make it it successful,
and highlight recent development to better serve new use cases
and future improvements.
% - 2-3 sentences putting the main findings into general context so it is
%   clear how the results described in the paper have moved the field forwards.
NumPy continues to evolve to meet the needs of its community, to remain
relevant in an ever-changing computing landscape, and to be a fundamental
pillar of support for the next decade of scientific computing, data science,
and machine learning in Python.
