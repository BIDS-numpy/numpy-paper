% https://www.nature.com/nature/for-authors/formatting-guide

% Articles start with a fully referenced summary paragraph, ideally of no more
% than 200 words, which is separate from the main text and avoids numbers,
% abbreviations, acronyms or measurements unless essential. It is aimed at
% readers outside the discipline.

% This summary paragraph should be structured as follows:
% - 2-3 sentences of basic-level introduction to the field;
% - a brief account of the background and rationale of the work;
% - a statement of the main conclusions (introduced by the phrase 'Here we show' or its equivalent);
% - 2-3 sentences putting the main findings into general context so it is
%   clear how the results described in the paper have moved the field forwards.

% Please refer to our annotated example  to see how the summary paragraph should
% be constructed. https://www.nature.com/documents/nature-summary-paragraph.pdf


% - 2-3 sentences of basic-level introduction to the field;
Array programming provides a powerful, compact, expressive syntax for accessing,
manipulating, and operating on data in vectors, matrices, and
higher-dimensional arrays \cite{iverson1980notation}.
% - a brief account of the background and rationale of the work;
NumPy is the fundamental array programming library for the Python language
\cite{dubois2007guest,oliphant2007python,millman2011python,perez2011python}.
Developed for over two decades by more than 850 contributors, 
it has nearly 300,000 dependent repositories on GitHub and millions of
downloads per year.
%It is one of the most commonly used machine learning tools among
%enterprises \cite{451report2018}.
It plays an essential role in research analysis pipelines in fields as
diverse as physics, chemistry, astronomy, geoscience, biology, psychology,
material science, engineering, finance, and economics.
For example, in astronomy, NumPy is an important part of the data analysis
pipeline used in the discovery of gravitational
waves \cite{abbott2016observation}, the first imaging of a black hole \cite{eht-imaging}, and
the continuing exploration of our universe \cite{jenness2018lsst}.
% - a statement of the main conclusions (introduced by the phrase 'Here we show' or its equivalent);
Here we show how a few fundamental concepts lead to a simple and
powerful programming paradigm for organizing, exploring, and analyzing
scientific data.
% - 2-3 sentences putting the main findings into general context so it is
%   clear how the results described in the paper have moved the field forwards.
NumPy is the foundation upon which the entire scientific Python
universe is constructed. It is so pervasive that several projects,
targeting audiences with specialized needs, have developed their own
NumPy-like interfaces and array objects.  Because of its central position in the
ecosystem, NumPy increasingly is used to bridge disparate
technologies, opening new possibilities in previously unexplored computational frontiers.

%Because of this, NumPy plays a unique role in enabling disparate technologies
%to interact and exchange data.
%NumPy continues to evolve to meet the needs of its community, to remain
%relevant in an ever-changing computing landscape, and to be a fundamental
%pillar of support for the next decade of scientific computing, data science,
%and machine learning in Python.
