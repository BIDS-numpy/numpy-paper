% https://www.nature.com/nature/for-authors/formatting-guide

% Articles start with a fully referenced summary paragraph, ideally of no more
% than 200 words, which is separate from the main text and avoids numbers,
% abbreviations, acronyms or measurements unless essential. It is aimed at
% readers outside the discipline.

% This summary paragraph should be structured as follows:
% - 2-3 sentences of basic-level introduction to the field;
% - a brief account of the background and rationale of the work;
% - a statement of the main conclusions (introduced by the phrase 'Here we show' or its equivalent);
% - 2-3 sentences putting the main findings into general context so it is
%   clear how the results described in the paper have moved the field forwards.

% Please refer to our annotated example  to see how the summary paragraph should
% be constructed. https://www.nature.com/documents/nature-summary-paragraph.pdf

NumPy is the fundamental array computation library for the Python ecosystem
\cite{dubois2007guest,oliphant2007python,millman2011python,perez2011python}.
It has been actively developed for over two decades and has
more than 850 contributors, nearly 300,000 dependent repositories
on GitHub, and millions of downloads per year.
It is extensively used in research and teaching.
According to a 2018 survey, NumPy is the second most commonly used machine
learning tool among enterprises \cite{451report2018}.
NumPy continues to evolve to meet the needs of its community, to
remain relevant in an ever-changing computing landscape, and to be a
fundamental pillar of support for the next decade of data science in
Python.
In this article, we provide an overview of the project and library
organization,  highlight some recent technical developments, and
outline our future plans.
