\documentclass[fleqn,10pt,twocolumn]{wlscirep}

% Packages
\usepackage{pifont}
\usepackage{amsfonts}
\usepackage{amsmath}
\usepackage{float}
% \usepackage[multiple]{footmisc}
\usepackage{booktabs}
\usepackage{csquotes}
\usepackage{moreverb} % for verbatim ouput (for wordcount)
\usepackage{listings, textcomp}
\lstset{language=Python}

\lstset{ %
  basicstyle=\ttfamily\footnotesize,  % size of fonts used for the code
  xleftmargin=\parindent,
  breaklines=true,   % automatic line breaking only at whitespace
  captionpos=b,   % sets the caption-position to bottom
  commentstyle=\color{gray},  % comment style
  keywordstyle=\color{blue},  % keyword style
  stringstyle=\color{red},  % string literal style
  upquote=true  %straight single quotes (requires textcomp)
}

\usepackage{caption}
\captionsetup{%
   figurename=Fig.,
   tablename=Tab.
}
\usepackage{subcaption}

% New Commands
\newcommand{\cmark}{\ding{51}}%
\newcommand{\xmark}{\ding{55}}%
\newcommand{\code}[1]{\texttt{#1}}
\newcommand{\fixme}[1]{\textcolor{red}{{#1}}}
\newcommand{\inlinecite}[1]{\footnotesize\citen{#1}}
\newcommand{\tightlist}{\setlength{\itemsep}{0pt}\setlength{\parskip}{0pt}}

\usepackage[switch]{lineno}
\linenumbers

\title{NumPy---Array Programming in Python}

\author[1]{Charles R Harris}
\author[2]{Travis E. Oliphant}
\author[3,*]{St\'efan J. van der Walt}
\author[3,4,*]{K. Jarrod Millman}
\affil[1]{Independent Researcher}
\affil[2]{Quansight LLC, Austin, TX, USA}
\affil[3]{Berkeley Institute for Data Science, University of California, Berkeley, CA 94720, USA}
\affil[4]{Division of Biostatistics, University of California, Berkeley, CA 94720, USA}
\affil[*]{numpy.articles@gmail.com}


\keywords{Scientific computing, Python, Mathematics, Array Programming}

\begin{abstract}
% https://www.nature.com/nature/for-authors/formatting-guide

% Articles start with a fully referenced summary paragraph, ideally of no more
% than 200 words, which is separate from the main text and avoids numbers,
% abbreviations, acronyms or measurements unless essential. It is aimed at
% readers outside the discipline.

% This summary paragraph should be structured as follows:
% - 2-3 sentences of basic-level introduction to the field;
% - a brief account of the background and rationale of the work;
% - a statement of the main conclusions (introduced by the phrase 'Here we show' or its equivalent);
% - 2-3 sentences putting the main findings into general context so it is
%   clear how the results described in the paper have moved the field forwards.

% Please refer to our annotated example  to see how the summary paragraph should
% be constructed. https://www.nature.com/documents/nature-summary-paragraph.pdf


% - 2-3 sentences of basic-level introduction to the field;
Array programming provides powerful notation for accessing,
manipulating, and operating on data in vectors, matrices, and
higher-dimensional arrays \cite{iverson1980notation}.
% - a brief account of the background and rationale of the work;
NumPy is the fundamental array programming library for the Python ecosystem
\cite{dubois2007guest,oliphant2007python,millman2011python,perez2011python}.
It has been actively developed for over two decades and has more than 850
contributors, nearly 300,000 dependent repositories on GitHub, and millions of
downloads per year.
It plays an essential role in research analysis pipelines in fields as
diverse as physics, biology, astronomy, neuroscience, material science,
engineering, and chemistry.
For example, in astronomy, NumPy was fundamental to discovering gravitational
waves\cite{pycbc}, the first imaging of a black hole\cite{eht-imaging}, and
continues underpinning the future exploration of our
universe\cite{jenness2018lsst}.
It is one of the most commonly used machine learning tools among
enterprises \cite{451report2018}.
% - a statement of the main conclusions (introduced by the phrase 'Here we show' or its equivalent);
Here we show how a small group of students and scientists created this
popular tool, explain the core ideas that helped make it it successful,
and highlight recent development to better serve new use cases
and future improvements.
% - 2-3 sentences putting the main findings into general context so it is
%   clear how the results described in the paper have moved the field forwards.
NumPy continues to evolve to meet the needs of its community, to remain
relevant in an ever-changing computing landscape, and to be a fundamental
pillar of support for the next decade of scientific computing, data science,
and machine learning in Python.
 
\end{abstract}

\begin{document}

\flushbottom
\maketitle
\thispagestyle{empty}

% Count of words
\verbatiminput{wordcount.tex}

Two Python array packages existed before NumPy.
The Numeric package began in the mid-90s and provided an array object in
Python, written in C, and linking to standard fast implementations of linear
algebra.
Around 2000, the Space Telescope Science Institute (STScI) software group wrote
a reimplementation of much of Numeric, called NumArray, to support their work
on large memory-mapped arrays and arrays of mixed data type
records \cite{STScI-slither}.
This briefly caused the two communities to diverge, until
2005, when NumPy emerged as a ``best of both worlds'' unification of Numeric
and NumArray \cite{oliphant2006guide}.

Today, NumPy underpins almost every Python library that does scientific or
numerical computation including SciPy \cite{virtanen2019scipy},
Matplotlib \cite{hunter2007matplotlib}, pandas \cite{mckinney-proc-scipy-2010},
scikit-learn \cite{pedregosa2011scikit}, and
scikit-image \cite{vanderwalt2014scikit}.
It is a community developed, open source library, which provides a
multidimensional Python array object along with array-aware functions
that operate on it.
Because of its inherent simplicity, the NumPy array is
the {\it de facto} exchange format for array data in Python.
The library has such widespread adoption that not only the array object but also its
{\it Application Programming Interface} (API) has become ubiquitous as
a language for array programming.

\section*{NumPy arrays}

\begin{figure*}
  \centering
  \includegraphics[width=\textwidth]{static/sketches/array-concepts}   
  \caption{\textbf{Fundamental Array Concepts.}
    \textbf{a,} The NumPy array data structure and its associated metadata fields.
    \textbf{b,} Indexing an array with various types of arguments.
    \textbf{c,} Indexing with arrays, which broadcast the indexing arguments before performing the lookup.
    \textbf{d,} Broadcasting in scalar addition, and in the division of two-dimensional arrays.
    \textbf{e,} Reduction operations act along one or more axes. In this
    example, a three-dimensional array is shown to be summed along various single
    axes to produce two-dimensional results, or along two axes consecutively to
    produce a one-dimensional result.
   }
  \label{fig:array-concepts}
\end{figure*}

The NumPy array is a data structure that efficiently stores and accesses
multidimensional arrays \cite{vanderwalt2011numpy}, also known as tensors, that
enables a wide variety of scientific computation.
It consists of a pointer to memory, along with metadata used to interpret the
data stored there, notably {\em data type}, {\em shape}, and {\em strides}
(Fig.~\ref{fig:array-concepts}a).

The \emph{data type} describes the nature of elements stored in an array.
An array has a single data type, and each array element occupies the same
number of bytes in memory.
Examples of data types include real and complex numbers (of lower and higher
precision), strings, timestamps, and pointers to Python objects.

The \emph{shape} of an array determines the number of elements along each axis,
and the number of axes is the array's dimensionality.
For example, a vector of numbers can be stored as a one dimensional array of
shape $N$, while color videos are four dimensional arrays of shape
$(T, M, N, 3)$.

\emph{Strides} are necessary to interpret computer memory, which stores elements
linearly, as multidimensional arrays.
It describes the number of bytes to move forward in memory to jump from row to
row, column to column, and so forth.
Consider, for example, a 2-D array of floating point numbers with shape
$(4, 3)$, where each element occupies 8 bytes in memory.
To move between consecutive columns we need to jump forward 8 bytes in memory,
and to access the next row $3 \times 8 = 24$ bytes.
The strides of that array are therefore $(24, 8)$.  NumPy is able to
store arrays in either C or Fortran memory order, iterating
first over either rows or columns.  This allows external libraries
written in those languages to directly access NumPy array data in memory.

To manipulate the data structure in an intuitive manner, we provide users with
a simple yet powerful array expression syntax.
It supports operations such as indexing (selecting subarrays) and elementwise
calculations.
A high level syntax allows users to express themselves succinctly, while NumPy
deals with the underlying mechanics of making operations fast.

\emph{Indexing} an array returns single elements, subarrays, or elements that satisfy
a specific condition (Fig.~\ref{fig:array-concepts}b).
Arrays can even be indexed using other arrays (Fig.~\ref{fig:array-concepts}c).
Wherever possible, indexing that retrieves a subarray returns a {\em view} on
the original array, such that data is shared between the two arrays.
This provides a powerful way to operate on subsets of array data, while
limiting memory usage.

A powerful array feature that affects the way NumPy indexes and combines
arrays, is {\em broadcasting}.
When performing an elementwise operation (such as addition) on two arrays, one
expects that these arrays should have the same shape.
However, through broadcasting, NumPy allows the dimensions to differ, while
still producing results that appeal to intuition.
A trivial example is the addition of a scalar value to an array, but it also
generalizes to more complex examples such as scaling each column of an array,
or generating a grid of coordinates.
In broadcasting, one or both arrays are virtually duplicated (that is, without
copying any data in memory), so that the shapes of the operands match
(Fig.~\ref{fig:array-concepts}d).
Broadcasting is also applied whenever an array is indexed using arrays of
indices (Fig.~\ref{fig:array-concepts}c).

To complement the array syntax, NumPy includes functions that perform
elementwise calculations on arrays, including arithmetic, statistics, and
trigonometry.
They aim to loop over array elements near-optimally, taking into consideration,
e.g., strides, in order to best utilize the computer's fast cache memory.
Vectorized operations that would take many tens of lines to express in
languages such as C can often be implemented as a single, clear Python
expression.

Many of these functions, such as summation, support \emph{reductions}: aggregating
results across one or more dimensions of the array.
For example, summing a $n$-dimensional array over $d$ axes results in a
$(n-d)$-dimensional array (Fig.~\ref{fig:array-concepts}e).

Other array-aware functions include creating, reshaping, concatenating, and padding
arrays; searching, sorting and counting data; and reading and writing files.
NumPy provides extensive support for generating pseudorandom numbers and
includes an assortment of probability distributions.
It also performs accelerated linear algebra, utilizing one of several backends
such as OpenBLAS \cite{wang2013augem,xianyi2012model} and Intel MKL optimized for the CPUs at hand.

\section*{Supporting the ecosystem}

\fixme{Maybe some wording around the actual *support* element?}

% - review supporting the ecosystem (actual support element)
%   - docstrings / testing / packaging
%   - coordinating mechanism
%     - deprecation policy

Users predominantly interact with NumPy arrays using {\em indexing} (to access
subarrays or individual elements), {\em operators} (e.g., $+$, $-$, $\times$,
and, for matrix multiplication, $@$), as well as {\em array-aware functions};
together, these provide an easily readable, expressive, high-level API for
array programming.

Various libraries build on and extend this functionality to further enrich the
ecosystem of available tools.  For example, SciPy adds various mathematical,
scientific, and engineering routines; Pandas provides indexed arrays, known as
dataframes; xarray implements arrays with labeled axes; and Matplotlib
generates publication-ready figures and visualizations.  Employing these
low-level array-aware functions are widely used technique-specific libraries,
including scikit-learn (machine learning), scikit-image (image processing),
statsmodels (statistics),  networkx (graphs and complex networks)
\cite{SciPyProceedings_11}, and Napari (volumetric image visualization).

Finally, domain-specific libraries and applications are built on this diverse
foundation of tooling; these include Astropy (astronomy) \cite{astropy:2013,
astropy:2018}, Biopython (computational biology and bioinformatics) \cite{cock2009biopython},
NIPY (neuroimaging) \cite{millman2007analysis}, and Pangeo (earth science) \cite{2018EGUGA..2012146H}.

Exposing array programming primitives, as well as the surrounding ecosystem of
tools, in Python---an interpreted language---creates an ideal environment for
interactive, exploratory data analysis where users may iteratively inspect,
manipulate, and visualize their data \cite{perez2007ipython}.

\section*{The proliferation of arrays}

\fixme{Add references somewhere to Zarr, https://arrow.apache.org/,
https://parquet.apache.org/,
https://github.com/ruby-numo/numo-narray/wiki/Comparison-with-Numpy}

Recent years have seen a massive explosion in data science, machine learning,
and artificial intelligence.  NumPy and its API, underlying almost every tool
in the computational toolchain, has become ubiquitous.  Still, NumPy itself
could never hope to satisfy every single specialized need of the community; it
is limited in its ability to work with very large datasets, datasets split
across multiple systems, and computations that require specialized hardware.
Furthermore, pursuing solutions to these challenges is out of scope for NumPy
development.

The community's efforts to fill the resultant gaps therefore led to a
proliferation of arrays including distributed arrays---which are stored on
multiple computers, GPU arrays---which utilize graphics processing units for
fast computation, computation graphs---which postpone and combine calculations
before executing them efficiently, and sparse arrays---which typically contain
few non-zero values, and store only those in memory.
%distributed arrays, GPU arrays, delayed execution arrays, sparse arrays,
%foreign language arrays, and more.

This includes popular tensor computation libraries such as
CuPy\footnote{\url{https://cupy.chainer.org/}},
JAX\footnote{\url{https://jax.readthedocs.io/en/latest/jax.numpy.html}}, and
Apache MXNet\footnote{\url{https://numpy.mxnet.io/}}.
PyTorch\footnote{\url{https://pytorch.org/tutorials/beginner/blitz/tensor\_tutorial.html}}
and
Tensorflow\footnote{\url{https://www.tensorflow.org/tutorials/customization/basics}}
provide tensor APIs with NumPy-inspired semantics.

\section*{Interoperability and extensibility}

NumPy supports array operations between external array objects by
acting as a central coordination mechanism with a well-specified API.
With the proliferation of array implementations discussed above, a
widespread adoption of the NumPy API is valuable: it lowers the
barrier to entry for newcomers and provides the wider community with a
stable array programming interface. This, in turn, prevents disruptive
schisms like the divergence of Numeric and NumArray, by facilitating
the development of specialized solutions and new tools that operate in
concert.

Primarily, there
exist two types of Python array objects: (1) arrays that mimic NumPy arrays but are
fundamentally different and (2) subclasses of NumPy arrays.

There has been a steady increase in the number of external libraries that
provide arrays and a NumPy-like API for manipulating them.
Ideally, operating on these arrays using NumPy would simply work, so that end
users could write code once, and would then benefit from switching between
NumPy arrays, GPU arrays, distributed arrays, and so forth, as appropriate.

\begin{figure}
  \centering
  \includegraphics[width=.45\textwidth]{static/sketches/duck-arrays}
  \caption{\textbf{Interoperability.} \fixme{This figure needs work.}}\label{fig:duck-arrays}
\end{figure}


To facilitate \emph{interoperability}, the NumPy team has designed
``protocols'' (or contracts of operation), that allow for these arrays to be
passed to NumPy functions (Fig.~\ref{fig:duck-arrays}).
NumPy, in turn, dispatches operations to the originating library, as required.
% FIXME: expand to 3 or more sentences discussing current efforts and
% issues with those, how did adoption go, what is being done next
%\url{https://numpy.org/neps/nep-0037-array-module.html}
Versions of these protocols have been successfully deployed.
% Ralf: Could you provide some proof of effectiveness, e.g., "This has made it
% possible to create distributed GPU arrays, enabling .... [find a reference to
% work by Peter Entschev]".

% https://numpy.org/neps/nep-0016-abstract-array.html
% https://numpy.org/neps/nep-0018-array-function-protocol.html
% https://numpy.org/neps/nep-0022-ndarray-duck-typing-overview.html
% https://numpy.org/neps/nep-0030-duck-array-protocol.html
% https://github.com/numpy/numpy/blob/a111b551ae940d7d5f8523fef1cf3589c6ba00a0/doc/neps/nep-0033-extensible-dtypes.rst
% https://numpy.org/neps/nep-0037-array-module.html

Subclasses of NumPy often exist due to difficulty in constructing richer data
types, such as quantities with physical units \cite{astropy,Goldbaum2018,pint},
geometrical objects \cite{pygeos}, and missing numbers.
To improve \emph{extensibility}, we are currently overhauling the data type
system to make it easier to implement data types both in Python and C, and to
support these and other applications.

\section*{Discussion}

% aka why is this so successful
% some sense of ongoing work / future directions

NumPy was initially developed by students, faculty, and researchers to
provide a modern array programming library for Python.  These
user-developers frequently had to write code from scratch to solve
their or their colleagues' problems---often in low-level languages
that preceded Python, like Fortran \cite{dongarra2008netlib} and C.
To them, the advantages of an interactive, high-level array library
were evident. Their design of this new tool was informed by their
experiences with powerful interactive programming languages for
scientific computing such as APL \cite{iverson1962programming} and
Yorick \cite{munro1995using}, as well as commercial languages and
environments like IDL and Matlab.

NumPy has subsequently seen prolific adoption, with millions of users,
first from research and later also industry.  It spurred the
development of the larger scientific Python ecosystem and heralded the
current era of wide-spread use of Python for scientific computing. The
reasons for its success are certainly complex, but we argue that they
are broadly of \emph{practical}, \emph{philosophical},
\emph{social}, and \emph{technical} origins.

% Fortran / C
% A few sentences on the history of array programming, referencing APL and Fortran, are missing.

The \emph{practical} appeal of NumPy and its ecosystem is apparent
when you consider the following description of how APL users
work---one which
easily applies to scientific Python users too \cite{kromberg2007apl}:
\begin{quote}
Users of APL tend to ``live in their data''. Interactive interpreters allow them
to inspect, almost feel their  way forward, discovering successful snippets
of code through experimentation and collecting them
into imperative or functional programs according to taste.
\end{quote}
This modality of interactive scientific exploration would develop
further and eventually culminate in the Jupyter ecosystem of tools \cite{Kluyver:2016aa}.

NumPy also benefited from being written in Python, a
general-purpose language well-suited to standard programming tasks such as
cleaning data, interacting with web resources, and parsing text.
Adding fast array operations and linear algebra allows scientist to do all
their work with within a single language---and one that has the advantage of
being famously easy to learn and teach.  In fact, Python's adoption as
a primary learning language in many universities (e.g., at UC Berkeley
many lower division courses use Python including Foundations of Data Science,
Introduction to Computational Thinking with Data,
and The Structure and Interpretation of Computer Programs) have
further popularized NumPy as a tool of choice for modern data science.

As an open source project, NumPy also did not cost money, and was
unencumbered by license servers, dongles, and the likes.

In terms of \emph{philosophical} appeal, NumPy is open source
software.  So, of course, are many other packages that are no longer
relevant, and by itself it does not motivate NumPy's sucess. However,
many developers would never have participated in building a closed
project.  Open source software also appealed the many proponents of
reproducible science, who wanted a transparent, introspectable stack
of software underlying their results.

\emph{Socially}, there was a sense of building something
impactful together, for the benefit of many others.  Participating in
such an endeavor, within a welcoming community of like-minded
indivuals, held a powerful attraction for many early contributors.

Throughout its development, NumPy has upheld strong \emph{technical}
principles.  It is a community of practice that enshrines a culture of
employing software engineering practice to improve collaboration and
reduce error \cite{millman2014developing}.  This culture is not only
adopted by leaders in the project, but also enthusiastically taught to
newcomers. The NumPy team was early in adopting distributed revision
control and code review to improve collaboration on code, and
continuous testing that runs a large battery of automated tests for
every proposed change to NumPy.  The project has comprehensive,
high-quality documentation, integrated with the source
code \cite{vanderwalt2008scipy,harrington2008scipy,harrington2009scipy}.

% https://ras.ac.uk/sites/default/files/2020-01/Group%20Award%20-%20Astropy.pdf
% https://ras.ac.uk/news-and-press/news/leading-astronomers-and-geophysicists-honoured-ras-bicentenary-year-0

This culture of using best practices for producing reliable scientific software
has been eagerly adopted by the ecosystem of libraries that build on NumPy.
For example, in a recent award given by the Royal Astronomical Society to
Astropy, they state:
\begin{quotation}
\noindent\emph{The Astropy Project has provided hundreds of junior scientists
with experience in professional-standard software development practices
including use of version control, unit testing, code review and issue tracking
procedures. This is a vital skillset for modern researchers that is often
missing from formal university education in physics or astronomy.}
\end{quotation}
Community members explicitly work to address this lack of formal education
through formal courses and workshops
\cite{wilson-software-carpentry,hannay-scientific-software-survey,millman2018teaching}.
%\fixme{cite https://scipy-school.org/}.

Lastly, NumPy benefited greatly from the simplicity of its underlying data
structure, which made it a natural standard for exchanging array data between
libraries.  It also made it easy for other libraries to develop fast and
memory-efficient compiled code, usually in C or Fortran, that could manipulate
these arrays and pass them back to Python.

Over time the role of the NumPy has changed. It is no longer merely an
array library, but also a standard API for tensor computation and a
central coordinating mechanism between array types and technologies.
NumPy has become a fully-fledged array programming paradigm, that
continues to drive numerical exploration and computation in scientific
Python.

% S:
%
% Thinking about enabling factors, I'd characterize them into three categories:
%
% - Practical
% - Philosophical
% - Social
% - Technical
%
% E.g., practical: reason why we use Python (learn one language for
% everything); students don't have money, so want to avoid impracticalities of
% license dongles/servers, etc.
%
% Philosophical: science should be open, transparent; our software should be
% controlled by scientists, not designers that we don't have access to
%
% Social: joy of building these things together, friendly welcome into the
% community for many of us---appreciation of our work and hours


\bibliography{references}

\newpage

\section*{Methods}

We use Git for version control and GitHub as the public hosting service for our
official \emph{upstream} repository (\url{https://github.com/numpy/numpy}).
We each work in our own copy (or fork) of the project and use the
upstream repository as our integration point.
To get new code into the upstream repository, we use GitHub's
pull request (PR) mechanism.
This allows us to review code before integrating it as well as to run a
large number of tests on the modified code to ensure a bug free and stable
experience for our users.

We also use GitHub's issue tracking system to collect and triage bugs.

\subsection*{Library organization}

Broadly, the NumPy library consists of the following parts:
the NumPy array data structure \code{ndarray}; the so-called \emph{universal functions};
a set of library functions for manipulating arrays and doing scientific
computation; infrastructure libraries for unit tests and Python package
building; and the program \code{f2py} for wrapping Fortran code in Python \cite{peterson2009f2py}.
The \code{ndarray} and the universal functions are generally considered
the core of the library.
In the following, we give a brief summary of these components of the
library.

\paragraph{\emph{Core.}}  The \code{ndarray} data structure and the
universal functions make up the core of NumPy.

The \code{ndarray} is the data structure at the heart of NumPy.
The data structure stores regularly strided homogeneous data types
inside a contiguous block memory, allowing for the efficient representation
of $n$-dimensional data.
More details about the data structure are given in ``The NumPy array:
a structure for efficient numerical computation.''\cite{vanderwalt2011numpy}.

The \emph{universal functions}, or more concisely, \emph{ufuncs},
are functions written in C that implement efficient looping over
NumPy arrays. An important feature of ufuncs is the built-in
implementation of \emph{broadcasting}.  For example, the function
\code{arctan2(x, y)} is a ufunc that accepts two values and computes
$\tan^{-1}(y/x)$.  When arrays are passed in as the arguments,
the ufunc will take care of looping over the dimensions of the inputs
in such a way that if, say, \code{x} is a 1-D array with length 3, and
\code{y} is a 2-D array with shape $2 \times 1$, the output will be
an array with shape $2 \times 3$ (Fig.~\ref{fig:array-concepts}c).
The ufunc machinery takes care
of calling the function with all the appropriate combinations of
input array elements to complete the output array.
The elementary arithmetic operations of addition, multiplication, etc.,
are implemented as ufuncs, so that broadcasting also applies to expressions
such as \code{x + y * z}.

\paragraph{\emph{Computing libraries.}}
NumPy provides a large library of functions for array manipulation
and scientific computing, including functions for: creating, reshaping,
concatenating, and padding arrays; searching, sorting and counting data
in arrays; computing elementary statistics, such as the mean, median,
variance, and standard deviation; file I/O; and more.

A suite of functions for computing the \emph{fast Fourier transform (FFT)}
and its inverse is provided.

NumPy's linear algebra library includes functions for: solving linear
systems of equations; computing various functions of a matrix, including
the determinant, the norm, the inverse, and the pseudo-inverse;
computing the Cholesky, eigenvalue, and singular value decompositions of a matrix;
and more.

The random number generator library in NumPy provides alternative
\emph{bit stream generators} that provide the core function of generating
random integers.
A higher-level generator class that implements an assortment of
probability distributions is provided. It includes the beta, gamma
and Weibull distributions, the univariate and multivariate normal
distributions, and more.

\paragraph{\emph{Infrastructure libraries.}} NumPy provides utilities
for writing tests and for building Python packages.

The \code{testing} subpackage provides functions such as
\code{assert\_allclose(actual, desired)} that may be used in
test suites for code that uses NumPy arrays.

NumPy provides the subpackage \code{distutils} which includes functions and classes
to facilitate configuration, installation, and packaging of libraries depending on NumPy.
% Remove the PyPI reference entirely?
These can be used, for example, when publishing to the PyPI website.

\paragraph{\emph{F2PY.}}  The program \code{f2py} is a tool for
building NumPy-aware Python wrappers of Fortran functions.
NumPy itself does not use any Fortran code;  F2PY is part of NumPy
for historical reasons.


\subsection*{Governance}

% https://mail.python.org/pipermail/numpy-discussion/2015-October/073849.html
% https://github.com/numpy/numpy/pull/6352
NumPy adopted an official Governance Document on October~5,
2015\cite{NumPyProjectGovernance}.
Project decisions are usually made by consensus of interested contributors.
This means that, for most decisions, everyone is entrusted with veto power.
A Steering Council, currently composed of 12~members, facilitates this
process and oversees daily development of the project by contributing code
and reviewing contributions from the community.

% https://mail.python.org/pipermail/numpy-discussion/2018-July/078476.html
% https://github.com/numpy/numpy/pull/11865
NumPy's official Code of Conduct was approved on September~1, 2018\cite{NumPyCodeofConduct}.
In brief, we strive to:
\emph{be open};
\emph{be empathetic, welcoming, friendly, and patient};
\emph{be collaborative};
\emph{be inquisitive}; and
\emph{be careful in the words that we choose}.
The Code of Conduct also specifies how breaches can be reported and outlines
the process for responding to such reports.

\subsection*{Funding}

% BIDS -- UCB
% https://www.moore.org/grant-detail?grantId=GBMF5447
% $645,020 in 2016
% https://sloan.org/grant-detail/8222
% $659,359 in 2017
% https://bids.berkeley.edu/news/bids-receives-sloan-foundation-grant-contribute-numpy-development
% http://doi.org/10.5281/zenodo.3585761
% http://doi.org/10.5281/zenodo.3585767
In 2017, NumPy received its first large grants totaling 1.3M USD from the
Gordon \& Betty Moore and the Alfred P. Sloan foundations.
Stéfan van der Walt is the PI and manages four programmers working on the project.
These two grants focus on addressing the technical debt accrued over the years and
on setting in place standards and architecture to encourage more sustainable development.

% CZI -- NumFOCUS/QuanSight
% https://chanzuckerberg.com/eoss/proposals/strengthening-numpys-foundations-growing-beyond-code/
% NumPy and OpenBLAS received $195,000 in 2019
% https://labs.quansight.org/blog/2019/11/numpy-openblas-CZI-grant/
NumPy received a third grant for 195K USD from the Chan Zuckerberg
Initiative at the end of 2019 with Ralf Gommers as the PI.
This grant focuses on better serving NumPy's large number of beginning
to intermediate level users and on growing the community of NumPy
contributors.
It will also provide support to OpenBLAS, on which NumPy depends for
accelerated linear algebra.

Finally, since May 2019 the project receives around 10K USD annually from
Tidelift, a part of which is used to fund documentation and website
improvements.


\subsection*{Developers}

NumPy is currently maintained by a group of 23 contributors with commit rights
to the NumPy code base. Out of these, 17 maintainers were active in
2019, 4 of whom were paid to work on the project full-time.
Additionally, there are a few long term developers who contributed and maintain
specific parts of NumPy, but are not officially maintainers.

Over the course of its history, NumPy has attracted PRs by 823 contributors.
However, its development relies heavily on a small number
of active maintainers, who share more than half of the contributions among
themselves.

At a release cycle of about every half year, the five recent releases in the years
2018 and 2019 have averaged about 450~PRs each,\footnote{
    Note that before mid 2011, NumPy development did not happen on \url{github.com}.
    All data provided here is based on the development which happened through GitHub
    PRs. In some cases contributions by maintainers may not be categorized as such.}
% Since 1.14.0 (based on changelog): 381 + 438 + 490 + 531 + 402; last is preliminary
with each release attracting more than a hundred new contributors.
Figure~\ref{fig:prs-over-time} shows the number of PRs merged into the NumPy
master branch.
Although the number of PRs being merged fluctuates,
the plot indicates an increased number of contributions over the past
years.

\begin{figure}
    \centering
    \includegraphics[width=0.9\linewidth]{scripts/PRs-using-CURRENT_MAINTAINERS.pdf}
    \caption{Number of pull requests merged into the NumPy master branch for each
        quarter since 2012. The total number of PRs is indicated with the
        lower blue area showing the portion contributed by current or previous
        maintainers.}\label{fig:prs-over-time}
\end{figure}



\subsection*{Community calls}

The massive number of scientific Python packages that
built on NumPy meant that it had an unusually high need for stability.
So to guide our development we formalized the feature proposal process, and
constructed a development roadmap with extensive input and feedback from the
community.


Weekly community calls alternate between triage and
higher level discussion.  The calls not only involve developers from
the community, but provide a venue for vendors and other external
groups to provide input.  For example, after Intel produced a forked
version of NumPy, one of their developers joined a call to discuss
community concerns.

\subsection*{NumPy enhancement proposals}

Given the complexity of the codebase and the massive number of projects depending
on it, large changes require careful planning and substantial work.
NumPy Enhancement Proposals (NEPs) are modeled after
Python Enhancement Proposals (PEPs) for ``proposing major new
features, for collecting community input on an issue, and for
documenting the design decisions that have gone into
Python''\footnote{\url{https://numpy.org/neps/nep-0000.html}}.
Since then there have been 19 proposed NEPS---6 have been implemented,
4 have been accepted and are being implemented, 4 are under
consideration, 3 have been deferred or superseded, and 2 have been rejected
or withdrawn.


\subsection*{Central role}

NumPy plays a central role in building and standardizing much of the scientific
Python community infrastructure.
The docstring standard mentioned in the history section is now widely adopted.
We are also now using the NEP system as a way to help coordinate the larger
scientific Python community.
% https://numpy.org/neps/nep-0029-deprecation_policy.html
For example, in NEP 29, we recommend, along with leaders from various other
projects, that all projects across the Scientific Python ecosystem adopt a
common ``time window-based'' policy for support of Python and NumPy versions.
This standard will simplify downstream project and release planning.

\subsection*{Wheels build system}

A Python \emph{wheel}\cite{PEP427} is a standard file format for
distributing Python libraries.  In addition to Python code, a
wheel may include compiled C extensions and other binary data.
This is important, because many libraries, including NumPy,
require a C compiler and other build tools to build the software
from the source code, making it difficult for many users to install
the software on their own.  The introduction of wheels to the Python
packaging system has made it much easier for users to install
precompiled libraries.

A GitHub repository containing scripts to build NumPy wheels has
been configured so that a simple commit to the repository triggers
an automated build system that creates NumPy wheels for several
computer platforms, including Windows, Mac OSX and Linux.  The wheels
are uploaded to a public server and made available for anyone to use.
This system makes it easy for users to install precompiled versions
of NumPy on these platforms.

The technology that is used to build the wheels evolves continually.
At the time this paper is being written, a key component is the
\code{multibuild} suite of tools developed by Matthew Brett and
other developers\cite{multibuild}.  Currently, scripts using
\code{multibuild} are written for the continuous integration
platforms Travis-CI (for Linux and Mac OSX) and Appveyor
(for Windows).

\subsection*{Recent technical improvements}

With the recent infusion of funding and a clear process for coordinating with
the developer community, we have been able to tackle a number of important
large scale changes.
We highlight two of those below, as well as changes made to our testing
infrastructure to support hardware platforms used in large scale computing.

\subsection*{Array function protocol}

A vast number of projects are built on NumPy;
these projects are consumers of the NumPy API.
Over the last several years, a growing number of projects are providers of
a \emph{NumPy-like API} and array objects targeting audiences with specialized
needs beyond NumPy's capabilities.
For example, the NumPy API is implemented by several popular tensor computation
libraries including CuPy\footnote{\url{https://cupy.chainer.org/}},
JAX\footnote{\url{https://jax.readthedocs.io/en/latest/jax.numpy.html}},
and Apache MXNet\footnote{\url{https://numpy.mxnet.io/}}.
PyTorch\footnote{\url{https://pytorch.org/tutorials/beginner/blitz/tensor\_tutorial.html}}
and Tensorflow\footnote{\url{https://www.tensorflow.org/tutorials/customization/basics}}
provide tensor APIs with NumPy-inspired semantics.
It is also implemented in packages that support sparse arrays
such as \code{scipy.sparse} and \code{pydata.sparse}.
Another notable example is Dask, a library for parallel computing in
Python.  Dask adopts the NumPy API and therefore presents a familiar
interface to existing NumPy users, while adding powerful abilities to
parallelize and distribute tasks.

The multitude of specialized projects creates the difficulty that consumers
of these NumPy-like APIs write code specific to a single project and do not support
all of the above array providers.
This is a burden for users relying on the specialized array-like, since
a tool they need may not work for them.
It also creates challenges for end-users who need to transition
from NumPy to a more specialized array.
The growing multitude of specialized projects with NumPy-like APIs threatened
to again fracture the scientific Python community.

To address these issues NumPy has the goal of providing the fundamental
API for \emph{interoperability} between the various NumPy-like APIs.
An earlier step in this direction was the implementation of the
\code{\_\_array\_ufunc\_\_} protocol in NumPy 1.13, which enabled interoperability
for most mathematical functions.\cite{NEP13}
In 2019 this was expanded more generally with the inclusion of the
\code{\_\_array\_function\_\_} protocol into NumPy~1.17.
These two protocols allow providers of array objects to be interoperable
with the NumPy API: their arrays work correctly with almost all NumPy functions.\cite{NEP18}
For the users relying on specialized array projects it means that even though
much code is written specifically for NumPy arrays and uses the NumPy API as
\code{import numpy as np}, it can nevertheless work for them.
For example, here is how a CuPy GPU array can be passed through NumPy for
processing, with all operations being dispatched back to CuPy:

\begin{lstlisting}
import numpy as np
import cupy as cp

x_gpu = cp.array([1, 2, 3])
y = np.sum(x_gpu)  # Returns a GPU array
\end{lstlisting}

Similarly, user defined functions composed using NumPy can now be
applied to, e.g., multi-node distributed Dask arrays:

\begin{lstlisting}
import numpy as np
import dask.array as da


def f(x):
    """Function using NumPy API calls"""
    y = np.tensordot(x, x.T)
    return np.mean(np.log(y + 1))


x_local = np.random.random([10000, 10000])  # random local array
x_distr = da.random.random([10000, 10000])  # random distributed array

f(x_local)  # returns a NumPy array
f(x_distr)  # works, returns a Dask array
\end{lstlisting}

\subsection*{Random number generation}

The NumPy \code{random} module provides pseudorandom numbers from a wide range of
distributions. In legacy versions of NumPy, simulated random values are produced
by a \code{RandomState} object that: handles seeding and state initialization;
wraps the core pseudorandom number generator based on a Mersenne Twister
implementation\footnote{to be precise, the standard 32-bit version of MT19937};
interfaces with the underlying code that transforms random bits into
variates from other distributions; and supplies a singleton instance exposed in
the root of the random module.

The \code{RandomState} object makes a compatibility guarantee so that a fixed
seed and sequence of function calls produce the same set of values. This
guarantee has slowed progress since improving the underlying code requires
extending the API with additional keyword arguments. This guarantee continues to
apply to \code{RandomState}.

NumPy 1.17 introduced a new API for generating random numbers that use a more
flexible structure that can be extended by libraries or end-users. The new API
is built using components that separate the steps required to generate random
variates. Pseudorandom bits are generated by a bit generator. These bits are
then transformed into variates from complex distributions by a generator.
Finally, seeding is handled by an object that produces sequences of high-quality
initial values.

Bit generators are simple classes that manage the state of an underlying
pseudorandom number generator. NumPy ships with four bit generators. The default
bit generator is a 64-bit implementation of the Permuted Congruential Generator
\cite{pcg64} (\code{PCG64}). The three other bit generators are a 64-bit version
of the Philox generator\cite{random123} (\code{Philox}), Chris Doty-Humphrey's
Small Fast Chaotic generator\cite{practrand} (\code{SFC64}), and the 32-bit
Mersenne Twister\cite{mt19937} (\code{MT19937}) which has been used in older
versions of NumPy.\footnote{The
\href{https://github.com/bashtage/randomgen}{randomgen project} supplies a wide
range of alternative bit generators such as a cryptographic counter-based
generators (\code{AESCtr}) and generators that expose hardware random number
generators (\code{RDRAND})\cite{randomgen}.} Bit generators provide
functions, exposed both in Python and C, for generating random integer
and floating point numbers.

The \code{Generator} consumes one of the bit generators and produces variates
from complicated distributions. Many improved methods for generating random
variates from common distributions were implemented, including the Ziggurat
method for normal, exponential and gamma variates\cite{ziggurat}, and Lemire's
method for bounded random integer generation\cite{lemire}. The \code{Generator}
is more similar to the legacy \code{RandomState}, and its API is substantially
the same. The key differences all relate to state management, which has been
delegated to the bit generator. The \code{Generator} does not make the same
stream guarantee as the \code{RandomState} object, and so variates may differ
across versions as improved generation algorithms are
introduced.\footnote{Despite the removal of the compatibility guarantee, simple
reproducibility across versions is encouraged, and minor changes that do not
produce meaningful performance gains or fix underlying bug are not generally
adopted.}

Finally, a \code{SeedSequence} is used to initialize a bit generator. The seed
sequence can be initialized with no arguments, in which case it reads entropy
from a system-dependent provider, or with a user-provided seed. The seed
sequence then transforms the initial set of entropy into a sequence of
high-quality pseudorandom integers, which can be used to initialize multiple bit
generators deterministically. The key feature of a seed sequence is that
it can be used to spawn child \code{SeedSequence}s which allow to initialize
multiple distinct bit generators.
% Of course there is a diminishing chance of collisions...
% Only found http://www.pcg-random.org/posts/developing-a-seed_seq-alternative.html
% as a reference for seed sequence, it would be nice to cite something.
This capability allows a seed sequence to facilitate large distributed applications
where the number of workers required is not known. The sequences generated from
the same initial entropy and spawns are fully deterministic to ensure
reproducibility.

The three components are combined to construct a complete random number
generator.

\begin{lstlisting}
from numpy.random import (
    Generator,
    PCG64,
    SeedSequence,
)

seq = SeedSequence(1030424547444117993331016959)
pcg = PCG64(seq)
gen = Generator(pcg)
\end{lstlisting}

This approach retains access to the seed sequence which can then be
used to spawn additional generators.

\begin{lstlisting}
children = seq.spawn(2)
gen_0 = Generator(PCG64(children[0]))
gen_1 = Generator(PCG64(children[1]))
\end{lstlisting}

While this approach retains complete flexibility, the method
\code{np.random.default\_rng} can be used to instantiate a \code{Generator} when
reproducibility is not needed.

The final goal of the new API is to improve extensibility. \code{RandomState} is
a monolithic object that obscures all of the underlying state and functions. The
component architecture is one part of the extensibility improvements. The
underlying functions (written in C) which transform the output of a bit
generator to other distributions are available for use in CFFI. This allows the
same code to be run in both NumPy and dependent that can consume CFFI, e.g.,
Numba. Both the bit generators and the low-level functions can also be used in C
or Cython code.\footnote{As of 1.18.0, this scenario requires access to the
NumPy source. Alternative approaches that avoid this extra step are being
explored.}

\subsection*{Testing on multiple architectures}

At the time of writing the two fastest supercomputers in the
world, Summit and Sierra, both have IBM POWER9 architectures
\cite{top500nov2019}. In late 2018, Astra, the first ARM-based
supercomputer to enter the TOP500 list, went into production\cite{
astra-wiki}. Furthermore, over 100 billion ARM processors have been
produced as of 2017\cite{arm-architecture}, making it the most 
widely used instruction set architecture in the world.

Clearly there are motivations for a large scientific computing
software library to support POWER and ARM architectures. We've extended
our continuous integration (CI) testing to include \texttt{ppc64le}
(POWER8 on Travis CI) and ARMv8 (on Shippable service). We also test
with the s390x architecture (IBM Z CPUs on Travis CI) so that we
can probe the behavior of our library on a big-endian machine.
This satisfies one of the major components of
improved CI testing laid out in a version of our roadmap
\cite{numpy-roadmap}---specifically, ``CI for more exotic
platforms."

PEP 599\cite{PEP599} lays out a plan for new Python binary wheel
distribution support, \texttt{manylinux2014}, that adds
support for a number of architectures supported by the CentOS
Alternative Architecture Special Interest Group, including
ARMv8, ppc64le, as well as s390x. We are thus well-positioned
for a future where provision of binaries on these architectures
will be expected for a library at the base of the ecosystem.


\section*{Acknowledgments}

Jim Hugunin wrote Numeric in 1995, while a graduate student at MIT.
Hugunin based his package on previous work by Jim Fulton, then working at the
US Geological Survey, with input from many others.
After he graduated, Paul Dubois at the Lawrence Livermore National Laboratory
became the maintainer.
Many people contributed to the project including Travis Oliphant, David Ascher,
Tim Peters, and Konrad Hinsen.

Around 1998, the Space Telescope Science Institute (STScI) software group began
to use Python heavily, and, around 2000, wrote a reimplementation of
Numeric, called Numarray.
%https://lancesimms.com/programs/Python/pyraf/Docs/greenfield.pdf
Perry Greenfield created the original framework.
Many people contributed to it including Jay Todd Miller, JT Hsu, Richard
L White, Jochen Krupper, and Phil Hodge.

Eric Jones co-founded the SciPy community, gave early feedback on array
implementations, and provided funding and travel support to several community
members.
Numerous people contributed to the creation and growth of the larger SciPy
ecosystem, which gives NumPy much of its value.  Others have injected new
energy and ideas by creating experimental array packages.

Finally, we would like to thank the many members of the community who have provided
feedback, submitted bug reports, made improvements to the documentation,
code, or website, promoted its use in their scientific fields, and built
the vast ecosystem of tools and libraries around NumPy.
Ross Barnowski and Michael Eickenberg suggested text and provided
helpful feedback on the manuscript.

S.J.v.d.W and K.J.M. were funded in part by the Gordon and Betty Moore
Foundation through Grant GBMF3834 and by the Alfred P. Sloan Foundation through
Grant 2013-10-27 to the University of California, Berkeley.
S.J.v.d.W, S.B., M.P., and W.W. were funded in part by the Gordon
and Betty Moore Foundation through Grant GBMF5447 and by the Alfred
P. Sloan Foundation through Grant G-2017-9960 to the University of
California, Berkeley.

\section*{Author Contributions Statement}

K.J.M. and S.J.v.d.W composed the manuscript with input from others.
S.B., R.G., K.S., W.W., M.B., and T.J.R. contributed text.
All authors have contributed significant code, documentation, and/or expertise
to the NumPy project.
All authors reviewed the manuscript.

\section*{Competing Interests}

The authors declare no competing interests.

\end{document}
