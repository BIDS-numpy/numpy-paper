\documentclass[fleqn,10pt]{wlscirep}

% Packages
\usepackage{pifont}
\usepackage{amsfonts}
\usepackage{amsmath}
\usepackage{float}
% \usepackage[multiple]{footmisc}
\usepackage{booktabs}
\usepackage{csquotes}
\usepackage{listings, textcomp}

\lstset{ %
  basicstyle=\ttfamily\footnotesize,  % size of fonts used for the code
  breaklines=true,   % automatic line breaking only at whitespace
  captionpos=b,   % sets the caption-position to bottom
  commentstyle=\color{gray},  % comment style
  keywordstyle=\color{blue},  % keyword style
  stringstyle=\color{red},  % string literal style
  upquote=true  %straight single quotes (requires textcomp)
}

% New Commands
\newcommand{\cmark}{\ding{51}}%
\newcommand{\xmark}{\ding{55}}%
\newcommand{\code}[1]{\texttt{#1}}
\newcommand{\fixme}[1]{\textcolor{red}{{#1}}}
\newcommand{\inlinecite}[1]{\footnotesize\citen{#1}}

\usepackage{etoolbox}
\makeatletter
\patchcmd{\@maketitle}
 {\noindent
{\parbox{\dimexpr\linewidth-2\fboxsep\relax}{\color{color1}\large\sffamily\textbf{ABSTRACT}}}}
 {\noindent
{\parbox{\dimexpr\linewidth-2\fboxsep\relax}{\color{color1}\large\sffamily\textbf{ABSTRACT}}}}
 {}{}
\makeatother

\title{NumPy---Array Computation for Python}

\author[1]{Charles R. Harris}
\author[2,3,4,*]{K. Jarrod Millman}
\author[5,2,4,*]{St\'efan J. van der Walt}
\author[6,*]{Ralf Gommers}
\author[7]{Pauli Virtanen}
\author[8]{David Cournapeau}
\author[9]{Eric Wieser}
\author[10]{Julian Taylor}
\author[4]{Sebastian Berg}
\author[11]{Nathaniel J. Smith}
\author[12]{Robert Kern}
\author[4]{Matti Picus}
\author[13]{Stephan Hoyer}
\author[14]{Marten H. van Kerkwijk}
\author[2,15]{Matthew Brett}
\author[16]{Allan Haldane}
\author[17]{Jaime Fern\'andez del R\'io}
\author[18,19]{Mark Wiebe}
\author[6,20,21]{Pearu Peterson}
\author[22,23]{Pierre G\'erard-Marchant}
\author[24]{Kevin Sheppard}
\author[25]{Tyler Reddy}
\author[4]{Warren Weckesser}
\author[6]{Hameer Abbasi}
\author[26]{Christoph Gohlke}
\author[6]{Travis E. Oliphant}
\affil[1]{Independent Researcher, Logan, Utah, USA}
\affil[2]{Helen Wills Neuroscience Institute, University of California, Berkeley, Berkeley, CA, USA}
\affil[3]{Division of Biostatistics, University of California, Berkeley, Berkeley, CA, USA}
\affil[4]{Berkeley Institute for Data Science, University of California, Berkeley, Berkeley, CA, USA}
\affil[5]{Applied Mathematics, Stellenbosch University, Stellenbosch, South Africa}
\affil[6]{Quansight LLC, Austin, TX, USA}
\affil[7]{Department of Physics and Nanoscience Center, University of Jyv\"askyl\"a, Jyv\"askyl\"a, Finland}
\affil[8]{Mercari JP, Tokyo, Japan}
\affil[9]{Department of Engineering, University of Cambridge, Cambridge, UK}
\affil[10]{Independent Researcher, Karlsruhe, Germany}
\affil[11]{Independent Researcher, Berkeley, CA, USA}
\affil[12]{Enthought, Inc., Austin, TX, USA}
\affil[13]{Google Research, Mountain View, CA, USA}
\affil[14]{Department of Astronomy \& Astrophysics, University of Toronto, Toronto, ON, Canada}
\affil[15]{School of Psychology, University of Birmingham, Edgbaston, Birmigham, UK}
\affil[16]{Department of Physics, Temple University, Philadelphia, PA, USA}
\affil[17]{Google, Zurich, Switzerland}
\affil[18]{Department of Physics and Astronomy, The University of British Columbia, Vancouver, BC, Canada}
\affil[19]{Amazon, Seattle, Washington, USA}
\affil[20]{Independent Researcher, Saue, Estonia}
\affil[21]{Department of Mechanics and Applied Mathematics, Institute of Cybernetics at Tallinn Technical University, Tallinn, Estonia}
\affil[22]{Department of Biological and Agricultural Engineering, University of Georgia, Athens, GA}
\affil[23]{France-IX Services, Paris, France}
\affil[24]{Department of Economics, University of Oxford, Oxford, UK}
\affil[25]{CCS-7, Los Alamos National Laboratory, Los Alamos, NM, USA}
\affil[26]{Laboratory for Fluorescence Dynamics, Biomedical Engineering Department, University of California, Irvine, Irvine, CA, USA}
\affil[*]{millman@berkeley.edu, stefanv@berkeley.edu, ralf.gommers@gmail.com}


\keywords{Scientific computing, Python, Mathematics}

\begin{abstract}
  NumPy is the fundamental array computing library for Python.
  It has been actively developed for over two decades and has
  more than 850 contributors, nearly 300,000 dependent repositories
  on GitHub, and millions of downloads per year.
  It is extensively used in research and teaching.
  According to a 2018 survey, NumPy is the second most commonly used machine
  learning tool among enterprises \cite{451report2018}.
  In this article, we provide an overview of the capabilities and development
  practices of the NumPy library and highlight some recent technical
  developments.
\end{abstract}

\begin{document}

\flushbottom
\maketitle
\thispagestyle{empty}

\section*{Introduction}

NumPy\cite{vanderwalt2011numpy} is the fundamental array computation library for the Python
ecosystem \cite{dubois2007guest,oliphant2007python,millman2011python,perez2011python}.
It underpins almost every library that does numerical
computation, including SciPy.  NumPy consists of the `ndarray`-object
along with utility functions to operate on such arrays.  Because of
its inherent simplicity—being a pointer to memory with some
associated meta-data about shape, data-type, and so forth—the NumPy
array has become the {\it de facto} exchange format for array data in
Python.  That said, NumPy has now seen such widespread adoption that
not only the array object but also its {\it Application Programming
  Interface} (API) has become ubiquitous, and is imitated by, e.g.,
popular tensor computation libraries.

Over the past few years, with external financial support, NumPy has
seen renewed developer interest.  It is actively developed to meet the
needs of its community, including the addition of some fairly large
changes as outlined in Section XXX.

\section*{History}

NumPy was, at its inception, the unifier of the Python array
processing world.  Before NumPy, two packages—Numeric and
numarray—existed.  The first was developed by Paul DuBois and
colleagues at [the lab], and was aimed at fast manipulation of many
small arrays.  In contrast, to handle large astronomical images,
NumArray was developed [by STSCI?]. In other words, these array
objects had different strengths and weaknesses, and therefore tended
to divide the already small user community.  In 200x, Travis Oliphant
wrote NumPy, a ``best of both worlds'' unification of the two
projects, based on Numeric.  The arrival of stable version 1.0 in 200x unified the scientific Python
community, and heralded the era of popular scientific computing in
Python that has led to formidable successes such as [cite some impacts here].

While, by 2006, NumPy was already eminently usable, it lacked good
documentation. In order to use NumPy in his classes, Joe Harrington
from the University of South Florida sponsored several rounds of the
NumPy Documentation Marathon \cite{harrington2008scipy,vanderwalt2008scipy}.
During these intensive
documentation improvement sessions, a Wikipedia-like system was
developed to allow community members to contribute so-called
``docstrings''—text describing each NumPy function. A community
documentation standard was published so that these pieces of
documentation would all provide similar information, and conform to a
basic standard of quality. Experienced community members reviewed and
integrated contributed docstrings, and soon NumPy had thorough
documentation coverage.

For many years, NumPy operated without external funding, development
time mostly being contributed freely by students and researchers in
their spare time.  In-person meetings were sponsored off of
individual's research grants, and while industry sponsored
contributions were made, notably those by Mark Wiebe when he was
sponsored by Travis Oliphant at X, mostly the package had rapid growth
and adoption without external investment.  In 2014, NumPy received its
first large grants from the Moore and Sloan foundations, totaling 1.3M
dollars.  This allowed for hiring several full time programmers and,
with the support of community members, mobilized a project that had
become stagnant for the several years prior.

NumPy is now actively developed, with weekly online developer
meetings, a formalized NumPy enhancement proposal process, and
multiple active grants (but still almost no industry support).

For a more detailed history of the scientific Python ecosystem, please
refer to ``SciPy 1.0---Fundamental Algorithms for Scientific Computing
in Python'' \cite{virtanen2019scipy}.

\section*{Architecture and implementation choices}

\subsection*{Library organisation}
Broadly, the NumPy library consists of the following parts:
the NumPy array data structure; the so-called \emph{universal functions};
a set of library functions for manipulating arrays and doing scientific
computation; infrastructure libraries for unit tests and Python package
building; and the program \code{f2py} for wrapping Fortran code in Python.
The \code{ndarray} and the universal functions are generally considered
the core of the library.
In the following, we give a brief summary of these components of the
library.

\paragraph{\emph{Core.}}  The \code{ndarray} data structure and the
universal functions make up the core of NumPy.

The \code{ndarray} is the data structure at the heart of NumPy.
The data structure stores regularly strided homogeneous data types
in a contiguous block memory, allowing for the efficient representation
of $n$-dimensional data. \fixme{[How much detail do we want here?]}
More details about the data structure are given in ``The NumPy array:
a structure for efficient numerical computation.''\cite{vanderwalt2011numpy}.

The \emph{universal functions}, or more concisely, \emph{ufuncs},
are functions written in C that implement efficient looping over
NumPy arrays. An important feature of ufuncs is the built-in
implementation of \emph{broadcasting}.  For example, the function
\code{arctan2(x, y)} is a ufunc that accepts two values and computes
$\tan^{-1}(y/x)$.  When arrays are passed in as the arguments,
the ufunc will take care of looping over the dimensions of the inputs
in such a way that if, say, \code{x} is a 1-d array with length 3, and
\code{y} is a 2-d array with shape $2 \times 1$, the output will be
an array with shape $2 \times 3$.  The ufunc machinery takes care
of calling the function with all the appropriate combinations of
input array elements to complete the output array.
The elementary arithmetic operations of addition, multiplication, etc.,
are implemented as ufuncs, so broadcasting applies even in expressions
such as \code{x + y * z}.

\paragraph{\emph{Computing libraries.}}
NumPy provides a large library of functions for array manipulation
and scientific computing, including functions for: creating, reshaping,
concatenating, and padding arrays; searching, sorting and counting data
in arrays; computing elementary statistics, such as the mean, median,
variance, and standard deviation; file I/O; and more.

A suite of functions for computing the \emph{fast Fourier transform (FFT)}
and its inverse is provided.

NumPy's linear algebra library includes functions for: solving linear
systems of equations; computing various functions of a matrix, including
the determinant, the norm, the inverse, and the pseudo-inverse;
computing the Cholesky, eigenvalue, and singular value decompositions of a matrix;
and more.

The random number generator library in NumPy provides several alternative
\emph{bit stream generators} that provide the core function of generating
random integers.
A higher-level generator class that implements an assortment of
probability distributions is provided. It includes the beta, gamma
and Weibull distributions, the univariate and multivariate normal
distributions, and more.

A \emph{masked array} class, built on top of the \code{ndarray}, is
provided.  This class allows elements of an array to be masked,
indicating that the data is missing or unknown.

There is also a \code{matrix} class the implements 2-d arrays with
specialized behavior.  We mention it here for completeness, but the
use of \code{matrix} is discouraged.  The quirks of the \code{matrix}
class have proven to be troublesome, and with the addition of the
infix \code{@} operator to Python for matrix multiplication, there was no
longer a need for a separate \code{matrix} class, since one can
use a 2-d array instead.

\paragraph{\emph{Infrastructure libraries.}} NumPy provides utilities
for writing unit tests and for building Python packages.

The \code{testing} subpackage provides functions such as
\code{assert\_allclose(actual, desired)} that may be used in unit
test suites for code that uses NumPy arrays.

NumPy provides a subpackage includes functions and classes
that can be used by libraries that have NumPy as dependency,
to be used in the library's configuration code that installs the
library and that builds a package that can be used, for example,
on the PyPI web site.

\paragraph{\emph{F2PY.}}  The program \code{f2py} is a tool for
building NumPy-aware Python wrappers of Fortran functions.
NumPy itself does not use any Fortran code;  F2PY is part of NumPy
for historical reasons.

\subsection*{Common infrastructure}

\subsection*{Language choices}

\section*{Recent technical improvements}

\subsection*{NEPS revived and roadmap}

\subsection*{Random}

\subsection*{\_\_array\_func\_\_}

\subsection*{Testing on multiple architectures}

\section*{Project organization and community}

\subsection*{Governance}

\subsection*{Maintainers and contributors}

\subsection*{Funding}

\section*{Discussion}

% The Discussion should be succinct and must not contain subheadings.


\section*{Data Availability}
% The statement should be provided as a separate section (titled 'Data Availability') at the end of the main text, before the 'References' section.
All NumPy source code and most data generated for the current study are available in the NumPy GitHub repository, \url{https://github.com/numpy}. Some supporting code and data have also been stored in other public repositories cited by this manuscript.

\bibliography{references}

\section*{Acknowledgments}


\section*{Author Contributions Statement}



%Must include all authors, identified by initials, for example: A.A.
%conceived the experiment(s),  A.A. and B.A. conducted the experiment(s), C.A.
%and D.A. analysed the results.  All authors reviewed the manuscript.

\section*{Competing Interests}

The authors declare no competing interests.


%To include, in this order: \textbf{Accession codes} (where applicable);
%\textbf{Competing financial interests} (mandatory statement).

%The corresponding author is responsible for submitting a
%\href{http://www.nature.com/srep/policies/index.html#competing}{competing
%financial interests statement} on behalf of all authors of the paper. This
%statement must be included in the submitted article file.

\section*{Consortium}
\subsection*{NumPy Contributors}

{\bfseries
Adam Apple$^{5}$
}
\newline
\hfill \break
$^{5}$International Centre for Ecological Exploration, California, USA


\end{document}

%%% Local Variables:
%%% mode: latex
%%% TeX-master: t
%%% End:
