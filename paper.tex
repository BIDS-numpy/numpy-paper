\documentclass[fleqn,10pt]{wlscirep}

% Packages
\usepackage{pifont}
\usepackage{amsfonts}
\usepackage{amsmath}
\usepackage{float}
% \usepackage[multiple]{footmisc}
\usepackage{booktabs}
\usepackage{csquotes}
\usepackage{listings, textcomp}

\lstset{ %
  basicstyle=\ttfamily\footnotesize,  % size of fonts used for the code
  breaklines=true,   % automatic line breaking only at whitespace
  captionpos=b,   % sets the caption-position to bottom
  commentstyle=\color{gray},  % comment style
  keywordstyle=\color{blue},  % keyword style
  stringstyle=\color{red},  % string literal style
  upquote=true  %straight single quotes (requires textcomp)
}

% New Commands
\newcommand{\cmark}{\ding{51}}%
\newcommand{\xmark}{\ding{55}}%
\newcommand{\code}[1]{\texttt{#1}}
\newcommand{\fixme}[1]{\textcolor{red}{{#1}}}
\newcommand{\inlinecite}[1]{\footnotesize\citen{#1}}

\usepackage{etoolbox}
\makeatletter
\patchcmd{\@maketitle}
 {\noindent
{\parbox{\dimexpr\linewidth-2\fboxsep\relax}{\color{color1}\large\sffamily\textbf{ABSTRACT}}}}
 {\noindent
{\parbox{\dimexpr\linewidth-2\fboxsep\relax}{\color{color1}\large\sffamily\textbf{ABSTRACT}}}}
 {}{}
\makeatother

\title{NumPy---Array Computation for Python}

\author[1]{Charles R Harris}
\author[2]{Travis E. Oliphant}
\author[3,*]{St\'efan J. van der Walt}
\author[3,4,*]{K. Jarrod Millman}
\affil[1]{Independent Researcher}
\affil[2]{Quansight LLC, Austin, TX, USA}
\affil[3]{Berkeley Institute for Data Science, University of California, Berkeley, CA 94720, USA}
\affil[4]{Division of Biostatistics, University of California, Berkeley, CA 94720, USA}
\affil[*]{numpy.articles@gmail.com}


\keywords{Scientific computing, Python, Mathematics}

\begin{abstract}
  NumPy is the fundamental array computing library for Python.
  It has been actively developed for over two decades and has
  more than 850 contributors, nearly 300,000 dependent repositories
  on GitHub, and millions of downloads per year.
  It is extensively used in research and teaching.
  According to a 2018 survey, NumPy is the second most commonly used machine
  learning tool among enterprises \cite{451report2018}.
  In this article, we provide an overview of the capabilities and development
  practices of the NumPy library and highlight some recent technical
  developments.
\end{abstract}

\begin{document}

\flushbottom
\maketitle
\thispagestyle{empty}

\section*{Introduction}

NumPy is the fundamental array computation library for the Python
ecosystem.  It underpins almost every library that does numerical
computation, including SciPy.  NumPy consists of the `ndarray`-object
along with utility functions to operate on such arrays.  Because of
its inherent simplicity—being a pointer to memory with some
associated meta-data about shape, data-type, and so forth—the NumPy
array has become the {\it de facto} exchange format for array data in
Python.  That said, NumPy has now seen such widespread adoption that
not only the array object but also its {\it Application Programming
  Interface} (API) has become ubiquitous, and is imitated by, e.g.,
popular tensor computation libraries.

Over the past few years, with external financial support, NumPy has
seen renewed developer interest.  It is actively developed to meet the
needs of its community, including the addition of some fairly large
changes as outlined in Section XXX.

\section*{History}

NumPy was, at its inception, the unifier of the Python array
processing world.  Before NumPy, two packages—Numeric and
numarray—existed.  The first was developed by Paul DuBois and
colleagues at [the lab], and was aimed at fast manipulation of many
small arrays.  In contrast, to handle large astronomical images,
NumArray was developed [by STSCI?]. In other words, these array
objects had different strengths and weaknesses, and therefore tended
to divide the already small user community.  In 200x, Travis Oliphant
wrote NumPy, a ``best of both worlds'' unification of the two
projects, based on Numeric.  The arrival of stable version 1.0 in 200x unified the scientific Python
community, and heralded the era of popular scientific computing in
Python that has led to formidable successes such as [cite some impacts here].

While, by 2006, NumPy was already eminently usable, it lacked good
documentation. In order to use NumPy in his classes, Joe Harrington
from the University of South Florida sponsored several rounds of the
NumPy Documentation Marathon [citations].  During these intensive
documentation improvement sessions, a Wikipedia-like system was
developed to allow community members to contribute so-called
``docstrings''—text describing each NumPy function. A community
documentation standard was published so that these pieces of
documentation would all provide similar information, and conform to a
basic standard of quality. Experienced community members reviewed and
integrated contributed docstrings, and soon NumPy had thorough
documentation coverage.

For many years, NumPy operated without external funding, development
time mostly being contributed freely by students and researchers in
their spare time.  In-person meetings were sponsored off of
individual's research grants, and while industry sponsored
contributions were made, notably those by Mark Wiebe when he was
sponsored by Travis Oliphant at X, mostly the package had rapid growth
and adoption without external investment.  In 2014, NumPy received its
first large grants from the Moore and Sloan foundations, totaling 1.3M
dollars.  This allowed for hiring several full time programmers and,
with the support of community members, mobilized a project that had
become stagnant for the several years prior.

NumPy is now actively developed, with weekly online developer
meetings, a formalized NumPy enhancement proposal process, and
multiple active grants (but still almost no industry support).

For a more detailed history of the scientific Python ecosystem, please
refer to [SciPy arxiv / nature paper].

\section*{Project organization and community}

\subsection*{Governance}

\subsection*{Maintainers and contributors}

\subsection*{Funding}

\section*{Discussion}

% The Discussion should be succinct and must not contain subheadings.


\section*{Data Availability}
% The statement should be provided as a separate section (titled 'Data Availability') at the end of the main text, before the 'References' section.
All NumPy source code and most data generated for the current study are available in the NumPy GitHub repository, \url{https://github.com/numpy}. Some supporting code and data have also been stored in other public repositories cited by this manuscript.

\bibliography{references}

\section*{Acknowledgments}


\section*{Author Contributions Statement}



%Must include all authors, identified by initials, for example: A.A.
%conceived the experiment(s),  A.A. and B.A. conducted the experiment(s), C.A.
%and D.A. analysed the results.  All authors reviewed the manuscript.

\section*{Competing Interests}

The authors declare no competing interests.


%To include, in this order: \textbf{Accession codes} (where applicable);
%\textbf{Competing financial interests} (mandatory statement).

%The corresponding author is responsible for submitting a
%\href{http://www.nature.com/srep/policies/index.html#competing}{competing
%financial interests statement} on behalf of all authors of the paper. This
%statement must be included in the submitted article file.

\section*{Consortium}
\subsection*{NumPy Contributors}

{\bfseries
Adam Apple$^{5}$
}
\newline
\hfill \break
$^{5}$International Centre for Ecological Exploration, California, USA


\end{document}

%%% Local Variables:
%%% mode: latex
%%% TeX-master: t
%%% End:
