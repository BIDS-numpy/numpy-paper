\documentclass[fleqn,10pt]{wlscirep}

% Packages
\usepackage{pifont}
\usepackage{amsfonts}
\usepackage{amsmath}
\usepackage{float}
% \usepackage[multiple]{footmisc}
\usepackage{booktabs}
\usepackage{csquotes}
\usepackage{moreverb} % for verbatim ouput (for wordcount)
\usepackage{listings, textcomp}
\lstset{language=Python}

\lstset{ %
  basicstyle=\ttfamily\footnotesize,  % size of fonts used for the code
  breaklines=true,   % automatic line breaking only at whitespace
  captionpos=b,   % sets the caption-position to bottom
  commentstyle=\color{gray},  % comment style
  keywordstyle=\color{blue},  % keyword style
  stringstyle=\color{red},  % string literal style
  upquote=true  %straight single quotes (requires textcomp)
}

% New Commands
\newcommand{\cmark}{\ding{51}}%
\newcommand{\xmark}{\ding{55}}%
\newcommand{\code}[1]{\texttt{#1}}
\newcommand{\fixme}[1]{\textcolor{red}{{#1}}}
\newcommand{\inlinecite}[1]{\footnotesize\citen{#1}}
\newcommand{\tightlist}{\setlength{\itemsep}{0pt}\setlength{\parskip}{0pt}}


\usepackage{etoolbox}
\makeatletter
\patchcmd{\@maketitle}
 {\noindent
{\parbox{\dimexpr\linewidth-2\fboxsep\relax}{\color{color1}\large\sffamily\textbf{ABSTRACT}}}}
 {\noindent
{\parbox{\dimexpr\linewidth-2\fboxsep\relax}{\color{color1}\large\sffamily\textbf{ABSTRACT}}}}
 {}{}
\makeatother

\usepackage{lineno}
\linenumbers

\title{NumPy---Array Computation for Python}

\author[1]{Charles R Harris}
\author[2]{Travis E. Oliphant}
\author[3,*]{St\'efan J. van der Walt}
\author[3,4,*]{K. Jarrod Millman}
\affil[1]{Independent Researcher}
\affil[2]{Quansight LLC, Austin, TX, USA}
\affil[3]{Berkeley Institute for Data Science, University of California, Berkeley, CA 94720, USA}
\affil[4]{Division of Biostatistics, University of California, Berkeley, CA 94720, USA}
\affil[*]{numpy.articles@gmail.com}


\keywords{Scientific computing, Python, Mathematics}

\begin{abstract}
  NumPy is the fundamental array computing library for Python.
  It has been actively developed for over two decades and has
  more than 850 contributors, nearly 300,000 dependent repositories
  on GitHub, and millions of downloads per year.
  It is extensively used in research and teaching.
  According to a 2018 survey, NumPy is the second most commonly used machine
  learning tool among enterprises \cite{451report2018}.
  In this article, we provide an overview of the capabilities and development
  practices of the NumPy library and highlight some recent technical
  developments.
\end{abstract}


\begin{document}

\flushbottom
\maketitle
\thispagestyle{empty}

% Count of words
\verbatiminput{wordcount.tex}

\section*{Introduction}

NumPy is the fundamental array computation library for the Python
ecosystem \cite{dubois2007guest,oliphant2007python,millman2011python,perez2011python}.
It underpins almost every library that does numerical
computation, including SciPy.  NumPy consists of the \code{ndarray}-object
along with utility functions to operate on it.  Because of
its inherent simplicity—being a pointer to memory with some
associated meta-data about shape, data-type, and so forth—the NumPy
array is the {\it de facto} exchange format for array data in
Python.
NumPy, while having been wildly successful as a library and array exchange
project, has now also become a standard language for expressing array and
tensor computations.  
It has such widespread adoption that not only the array object but also its
{\it Application Programming Interface} (API) has become ubiquitous, and is
implemented by, for example, popular tensor computation libraries.

The arrival of a stable version of NumPy (1.0) in October 2006 unified
the scientific Python community, and heralded the era of popular
scientific computing in Python that has led to formidable successes
such as [cite some impacts here].

[LIGO, EHT, Higgs-Boson pipeline at CERN—double check this carefully,
  may be root, ice cube experiment, https://arxiv.org/abs/1507.03989
  IDL is now below Python, Pangeo <-> xarray, Earthcube,
  https://ai.jpl.nasa.gov/public/documents/papers/hackett\_spaceops2018\_block.pdf,
  Kepler, Mars rovers, SKA in South Africa, yt, astropy, seismo-live,
  icesat2 \& icepyx, The Materials Project at LBNL https://github.com/materialsproject]

\subsection*{History}

NumPy was, at its inception, the unifier of the Python array
processing world.  Before NumPy, two packages—Numeric and
numarray—existed.  The first was developed by Jim Fulton, Jim Hugunin, Paul Dubois, Konrad
Hinsen, Travis Oliphant, and others, and was aimed at fast manipulation of many
small arrays.  In contrast, to handle large astronomical images,
NumArray was developed at the Space Telescope Science Institute (STScI).
In other words, these array
objects had different strengths and weaknesses, and therefore tended
to divide the already small user community.  Furthermore, the divide
posed a challenge for SciPy, which built on Numeric and provided
fundamental scientific routines \cite{virtanen2019scipy}.
In response, the maintainer of Numeric, Travis Oliphant, started a
major rewrite that aimed to be a ``best of both worlds'' unification
of the two projects.  This became known as NumPy.

While, by 2006, NumPy was already eminently usable, it lacked good
documentation. In order to use NumPy and SciPy in his classes, Joe Harrington
from the University of South Florida sponsored several rounds of the
NumPy Documentation Marathon \cite{harrington2008scipy,vanderwalt2008scipy}.
During these intensive
documentation improvement sessions, we developed a Wikipedia-like
system for community members to contribute so-called
``docstrings''—text describing each NumPy function. A community
documentation standard was published so that these pieces of
documentation would all provide similar information, and conform to a
basic standard of quality. Experienced community members reviewed and
integrated contributed docstrings, and soon NumPy had thorough
documentation coverage.

To coordinate work as a global community, to ensure accuracy of computed
results, and to allow modification without breakage, the NumPy community
pioneered best software engineering practices in the scientific Python
community  \cite{millman2014developing}.
For example, NumPy early on adopted distributed revision control, unit testing,
continuous integration, code review, and standardized executable documentation.

However, after his initial effort to build NumPy, Travis Oliphant
stepped back from development to found a company and several projects,
including Numba---an accelerated Python compiler that could vastly
accelerate NumPy expressions.  By 2009, the project had entered a
period of uncoordinated, opportunistic development, and was
increasingly weighed down by technical debt that had accrued since the
days of Numeric.

\subsection*{Funding}

For many years, NumPy operated without external funding, development
time mostly being contributed freely by students and researchers in
their spare time.  In-person meetings were sponsored off of
individual's research grants, and while industry sponsored
contributions were made, notably those by Mark Wiebe when he was
sponsored by Travis Oliphant at Continuum Analytics, mostly the
package had rapid adoption and became central to scientific computing
without external investment.

% BIDS -- UCB
% https://www.moore.org/grant-detail?grantId=GBMF5447
% $645,020 in 2016
% https://sloan.org/grant-detail/8222
% $659,359 in 2017
% https://bids.berkeley.edu/news/bids-receives-sloan-foundation-grant-contribute-numpy-development

In 2017, the Berkeley Institute for Data Science
(BIDS) received grants totaling 1.3M USD from the Gordon \& Betty
Moore and the Alfred P. Sloan foundations to support the development
of NumPy for better data science.
Specifically, the aim was to address technical debt and set in place
standards and architecture to encourage more sustainable development.
Stéfan van der Walt, a senior research data scientist at BIDS and a
member of the NumPy steering council, is the PI and manages four
programmers working on the project.

% CZI -- NumFOCUS/QuanSight
% https://chanzuckerberg.com/eoss/proposals/strengthening-numpys-foundations-growing-beyond-code/
% NumPy and OpenBLAS received $195,000 in 2019
% https://labs.quansight.org/blog/2019/11/numpy-openblas-CZI-grant/

In 2019, NumFOCUS received a grant for \$195,000 in 2019 for the
development of NumPy and OpenBLAS (on which NumPy depends for
accelerated linear algebra) from the Chan Zuckerberg Initiative.
This grant, building on the technical work supported by the previous
grant, will focus on better serving NumPy's large number of beginner
to intermediate level users and on growing the community of NumPy
contributors.
Ralf Gommers, the Director of Quansight Labs and a member of the NumPy
steering council, is the PI.

The project receives around 10K USD per year from Tidelift, which is
used to fund documentation and website developers.

\subsection*{Current status}

% (maturity, users)

In 2007, the NumPy release manager, Jarrod Millman, adopted the notion
of enhancement proposals---used infrequently until 2017 when, to
facilitate work expected from funded development, he formalized the
NumPy Enhancement Proposal (NEP) process.  NEPs are modeled after
Python Enhancement Proposals (PEPs) for ``proposing major new
features, for collecting community input on an issue, and for
documenting the design decisions that have gone into
Python''\footnote{\url{https://numpy.org/neps/nep-0000.html}}.
Since then there have been 19 proposed NEPS---6 have been implemented,
4 have been accepted and are being implemented, 4 are under
consideration, 3 have been deferred or superseded, and 2 have been rejected
or withdrawn.

The first activities organized for the new grant were a NEP development
sprint, a weekly community call, and a meeting to write a draft roadmap.

Since the roadmap was meant to represent community consensus on future
development, it was presented for comment at the annual SciPy
conference during a so-called ``Birds of a Feather'' session, attended
by more than 100 people.  One additional meeting was held to discuss
specific enhancements, such as the development of a new data-type
system.  The final roadmap was vetted by the community via online
discussion before being accepted.

The weekly community calls alternate weekly between triage meetings and
higher level discussion.  The calls not only involve developers from
the community, but provide a venue for vendors and other external
groups to provide input.  For example, after Intel produced a forked
version of NumPy, one of their developers joined a call to discuss
community concerns.

NumPy plays a central role in building and standardizing much of the scientific
Python community infrastructure.
The docstring standard mentioned in the history section is now widely adopted.
We are also now using the NEP system as a way to help coordinate the larger
scientific Python community.
% https://numpy.org/neps/nep-0029-deprecation_policy.html
For example, in NEP 29, we recommend, along with leaders from various other
projects, that all projects across the Scientific Python ecosystem adopt a
common ``time window-based'' policy for support of Python and NumPy versions.
This standard will simplify downstream project and release planning.

% https://numpy.org/neps/scope.html
% https://numpy.org/neps/roadmap.html

We showed that NumPy is actively developed according to a community
vetted plan that integrates formal enhancement proposals.  Next, we
give a more in-depth overview of the team developing NumPy, their
activities, as well as the governance and community structures that
guide the collaboration.

\section*{Project organization and community}

NumPy is currently maintained by a group of 23 contributors with commit rights
to the NumPy code base. Out of these 17 maintainers were active in 2019.
Additionally, there are a few long term developers who contributed and maintain
specific parts of NumPy, but are not officially maintainers.
At a release cycle of about every half year, the five recent releases in the years
2018 and 2019 have averaged about 450~pull requests~(PR) each.\footnote{
    Note that before mid 2011, NumPy development did not happen on \url{github.com}.
    All data provided here is based on the development which happened through github
    pull requests. In some cases contributions by maintainers may not be categorized as such.}
% Since 1.14.0 (based on changelog): 381 + 438 + 490 + 531 + 402; last is preliminary
With each release attracting more than a hundred new contributors.

\begin{figure}
    \centering
    \includegraphics[width=0.5\textwidth]{scripts/PRs-using-CURRENT_MAINTAINERS.pdf}
    \caption{Number of pull requests merged into the NumPy master branch for each
        quarter since 2012. The total number of PR is indicated with the
        lower blue area showing the portion contributed by current or previous
        maintainers.}\label{fig:prs-over-time}
\end{figure}

The stacked plot in Figure~\ref{fig:prs-over-time} shows the number of
PR merged into the NumPy master branch.
In this plot PR opened by current and old maintainers are indicated.
Although the number of PRs being merged fluctuates,
the plot indicates an increased number of contributions over the past
years.
Over the course of its history, NumPy has attracted PR by 823 contributors.
However, its development still relies heavily on a relatively small number
of active maintainers, who share more than half of the contributions among
themselves.

% https://mail.python.org/pipermail/numpy-discussion/2015-October/073849.html
% https://github.com/numpy/numpy/pull/6352
NumPy adopted an official Governance Document on October~5,
2015\cite{NumPyProjectGovernance}.
Project decisions are usually made by consensus of interested contributors.
This means for many thing everyone is entrusted with veto power.
A Steering Council, currently composed of 12~members, facilitates this
process and oversees daily development of the project by contributing code
and reviewing contributions from the community.

% https://mail.python.org/pipermail/numpy-discussion/2018-July/078476.html
% https://github.com/numpy/numpy/pull/11865
NumPy's official Code of Conduct was approved on September~1, 2018\cite{NumPyCodeofConduct}.
In brief, we strive to:
\emph{be open};
\emph{be empathetic, welcoming, friendly, and patient};
\emph{be collaborative};
\emph{be inquisitive}; and
\emph{be careful in the words that we choose}.
The Code of Conduct also specifies how breaches can be reported and outlines
the process for responding to such reports.

% \subsection*{GSoD}


\section*{Library organization}

The NumPy array data structure and universal functions, which we describe
below, make up the core of NumPy.
The NumPy library also includes
a set of library functions for manipulating arrays and doing scientific computation,
libraries for unit tests and Python package building,
and the program \code{f2py}\cite{peterson2009f2py} for wrapping Fortran code in Python.

\subsection*{Core}

The \code{ndarray} data structure stores regularly strided homogeneous data
types in a contiguous block memory, allowing for the efficient representation
of $n$-dimensional data.
A \emph{masked array} class, built on top of the \code{ndarray}, is
provided as well. This class allows elements of an array to be masked,
indicating that the data is missing or unknown.
More details about the data structure are given in ``The NumPy array:
a structure for efficient numerical computation.''\cite{vanderwalt2011numpy}.

The \emph{universal functions} (or \emph{ufuncs})
are functions written in C that implement efficient looping over
NumPy arrays. An important feature of ufuncs is the built-in
implementation of \emph{broadcasting}.  For example, the function
\code{arctan2(x, y)} is a ufunc that accepts two values and computes
$\tan^{-1}(y/x)$.  When arrays are passed in as the arguments,
the ufunc will take care of looping over the dimensions of the inputs
in such a way that if, say, \code{x} is a 1-d array with length 3, and
\code{y} is a 2-d array with shape $2 \times 1$, the output will be
an array with shape $2 \times 3$.  The ufunc machinery takes care
of calling the function with all the appropriate combinations of
input array elements to complete the output array.
The elementary arithmetic operations of addition, multiplication, etc.,
are implemented as ufuncs, so broadcasting applies even in expressions
such as \code{x + y * z}.

\subsection*{Computing libraries}

NumPy provides a large library of functions for array manipulation
including functions for: creating, reshaping, concatenating, and padding arrays;
searching, sorting and counting data
in arrays; computing elementary statistics, such as the mean, median,
variance, and standard deviation; file I/O; and more.
For historical reasons, NumPy also includes basic functionality for
linear algebra,
fast Fourier transforms and windowing,
and polynomial fitting.
We will keep supporting these, but we will not expand beyond them.


% use scipy for linear algebra etc...

%A suite of functions for computing the \emph{fast Fourier transform (FFT)}
%and its inverse is provided.

%NumPy's linear algebra library includes functions for: solving linear
%systems of equations; computing various functions of a matrix, including
%the determinant, the norm, the inverse, and the pseudo-inverse;
%computing the Cholesky, eigenvalue, and singular value decompositions of a matrix;
%and more.
%

%For historical reasons, we provide a \code{matrix} class the implements 2-d
%arrays with specialized behavior.
%But we strongly recommend you use a 2-d array and use the infix \code{@} operator
%for matrix multiplication.
%
%There is also a \code{matrix} class the implements 2-d arrays with
%specialized behavior.  We mention it here for completeness, but the
%use of \code{matrix} is discouraged.  The quirks of the \code{matrix}
%class have proven to be troublesome, and with the addition of the
%infix \code{@} operator to Python for matrix multiplication, there was no
%longer a need for a separate \code{matrix} class, since one can
%use a 2-d array instead.

\subsection*{Infrastructure libraries}

Being at the base of the scientific Python stack, NumPy provides some basic
infrastructure for other packages in the scientific Python ecosystem.
The \code{numpy.testing} subpackage provides functions such as
\code{assert\_allclose(actual, desired)} that may be used in unit
test suites for code that uses NumPy arrays.
The \code{numpy.distutils} subpackage provides build support for C++, Fortran,
BLAS/LAPACK, and other relevant libraries for scientific computing.
The program \code{f2py} is a tool for
building NumPy-aware Python wrappers of Fortran functions.
NumPy itself does not use any Fortran code;  F2PY is part of NumPy
for historical reasons.


% \section*{Common infrastructure}

%\subsection*{Wheels build system}
%
%A Python \emph{wheel}\cite{PEP427} is a standard file format for
%distributing Python libraries.  In addition to Python code, a
%wheel may include compiled C extensions and other binary data.
%This is important, because many libraries, including NumPy,
%require a C compiler and other build tools to build the software
%from the source code, making it difficult for many users to install
%the software on their own.  The introduction of wheels to the Python
%packaging system has made it much easier for users to install
%precompiled libraries.
%
%A GitHub repository containing scripts to build NumPy wheels has
%been configured so that a simple commit to the repository triggers
%an automated build system that creates NumPy wheels for several
%computer platforms, including Windows, Mac OSX and Linux.  The wheels
%are uploaded to a public server and made available for anyone to use.
%This system makes it easy for users to install precompiled versions
%of NumPy on these platforms.
%
%The technology that is used to build the wheels evolves continually.
%At the time this paper is being written, a key component is the
%\code{multibuild} suite of tools developed by Matthew Brett and
%other developers\cite{multibuild}.  Currently, scripts using
%\code{multibuild} are written for the continuous integration
%platforms Travis-CI (for Linux and Mac OSX) and Appveyor
%(for Windows).


\section*{Recent technical improvements}

NumPy has, throughout its lifetime, seen continuous development.
E.g., bugs fixes, minor improvements, cleanups, and releases have
happened regularly.  However, with the NEP process in place we now
have the flexibility to plan and implement changes of larger scope.
We highlight two of those below, as well as changes made to our
testing infrastructure to support hardware platforms used in large
scale computing.

\subsection*{The Array Function Protocol}

The central position of NumPy takes within the numerical and scientific
Python ecosystem means that a vast number of projects are built on it.
These projects are consumers of the NumPy API.

On the other side, projects are providers of a \emph{NumPy-like API} and array
objects targeting audiences with specialized needs beyond NumPy's capabilities.
For example, the NumPy API is implemented by
several popular tensor computation libraries including
CuPy\footnote{\url{https://cupy.chainer.org/}},
Jax\footnote{\url{https://jax.readthedocs.io/en/latest/jax.numpy.html}},
PyTorch\footnote{\url{https://pytorch.org/tutorials/beginner/blitz/tensor\_tutorial.html}}, and
Apache MXNet\footnote{\url{https://mxnet.incubator.apache.org/api/python/docs/api/ndarray/index.html}},
and to a lesser extent
TensorFlow\footnote{\url{https://www.tensorflow.org/api\_docs/python/tf/Tensor}}.
It is also implemented in packages that support sparse arrays
such as \code{scipy.sparse} and \code{pydata.sparse}.
Another notable example is Dask, a library for parallel computing in
Python.  Dask adopts the NumPy API and therefore presents a familiar
interface to existing NumPy users, while adding powerful abilities to
parallelize and distribute tasks.

The multitude of specialized projects creates the difficulty that consumers
of these NumPy-like APIs write code specific to a single project and do not support
all of the above array providers.
This is a burden for users relying on the specialized Array-like, since
a tool they need may not work for them.
Additionally, a user who has the need of transitioning from NumPy to a more
specialized array, is faced with a challenge.
These issue can lead into a fragmentation of the user bases into consumers
of different NumPy-like APIs.

To address these issues NumPy has the goal of providing the fundamental
API for \emph{interoperability} between the various NumPy-like APIs.
An earlier step in this direction was the implementation of the
\code{\_\_array\_ufunc\_\_} protocol in NumPy 1.13, which enabled interoperability
for most mathematical functions.\cite{NEP13}
In 2019 this was expanded more generally with the inclusion of the
\code{\_\_array\_function\_\_} protocol into NumPy~1.17.
These two protocols allows providers of Array-like APIs to be interoperable
with the NumPy API: their arrays work correctly with almost all NumPy functions.\cite{NEP18}
For the users relying on specialized array projects it means that even though
much code is written specifically for NumPy arrays and uses the NumPy API as
\code{import numpy as np}, it can nevertheless work for them.  For
example, here is how a CuPy GPU array can be passed through NumPy for
processing, with all operations being dispatched back to CuPy:

\begin{lstlisting}
  import numpy as np
  import cupy as cp

  x_gpu = cp.array([1, 2, 3])
  y = np.sum(x_gpu)  # This works, and returns a GPU array
\end{lstlisting}

Similarly, user defined functions composed using NumPy can now be
applied to, e.g., multi-node distributed Dask arrays:

\begin{lstlisting}
  import numpy as np
  import dask.array as da


  def f(x):
      """Function using NumPy API calls"""
      y = np.tensordot(x, x.T)
      return np.mean(np.log(y + 1))


  x_local = np.random.random([10000, 10000])  # random local array
  x_distr = da.random.random([10000, 10000])  # random distributed array

  f(x_local)  # returns a NumPy array
  f(x_distr)  # works, returns a Dask array
\end{lstlisting}

\subsection*{Random}

%The random number generator library in NumPy provides several alternative
%\emph{bit stream generators} that provide the core function of generating
%random integers.
%A higher-level generator class that implements an assortment of
%probability distributions is provided. It includes the beta, gamma
%and Weibull distributions, the univariate and multivariate normal
%distributions, and more.

The NumPy random module provides pseudorandom numbers from a wide range of
distributions. In legacy versions of NumPy, simulated random values are produced
by a \code{RandomState} object that: handles seeding and state initialization;
wraps the core pseudorandom number generator based on a 32-bit implementation of
MT19937; interfaces with the underlying code that transforms random bits into
deviates from other distributions; and supplies a singleton instance exposed in
the root of the random module.

The \code{RandomState} object makes a compatibility guarantee so that a fixed
seed and sequence of function calls produce the same set of values. This
guarantee has slowed progress since improving the underlying code requires
extending the API with additional keyword arguments. This guarantee continues to
apply to \code{RandomState}. 

% Mention explicitly the NEPs involved

NumPy 1.17 introduced a new API for generating random numbers that use a more
flexible structure that can be extended by libraries or end-users. The new API
is built using components that separate the steps required to generate random
variates. Pseudorandom bits are generated by a bit generator. These bits are
then transformed into variates from complex distributions by a generator.
Finally, seeding is handled by an object that produces sequences of high-quality
initial values.

Bit generators are simple classes that manage the state of an underlying
pseudorandom number generator. NumPy ships with four bit generators. The default
bit generator is a 64-bit implementation of the Permuted Congruential Generator
\cite{pcg64} (\code{PCG64}). The three other bit generators are a 64-bit version
of the Philox generator\cite{random123} (\code{Philox}), Chris Doty-Humphrey's
Small Fast Chaotic generator\cite{practrand} (\code{SFC64}), and the 32-bit
Mersenne Twister\cite{mt19937} (\code{MT19937}) which has been used in older
versions of NumPy.\footnote{The
\href{https://github.com/bashtage/randomgen}{randomgen project} supplies a wide
range of alternative bit generators such as a cryptographic counter-based
generators (\code{AESCtr}) and generators that expose hardware random number
generators (\code{RDRAND})\cite{randomgen}.} Bit generators export a capsule
containing pointers to functions that produce 64 bits, 32 bits, or a random
double. They also expose CFFI and ctypes interfaces to the same three functions.

The \code{Generator} consumes one of the bit generators and produces variates
from complicated distributions. Many improved methods for generating random
variates from common distributions were implemented, including the Ziggurat
method for normal, exponential and gamma variates\cite{ziggurat}, and Lemire's
method for bounded random integer generation\cite{lemire}. The \code{Generator}
is more similar to the legacy \code{RandomState}, and its API is substantially
the same. The key differences all relate to state management, which has been
delegated to the bit generator. The \code{Generator} does not make the same
stream guarantee as the \code{RandomState} object, and so variates may differ
across versions as improved generation algorithms are
introduced.\footnote{Despite the removal of the compatibility guarantee, simple
reproducibility across versions is encouraged, and minor changes that do not
produce meaningful performance gains or fix underlying bug are not generally
adopted.}

Finally, a \code{SeedSequence} is used to initialize a bit generator. The seed
sequence can be initialized with no arguments, in which case it reads entropy
from a system-dependent provider, or with a user-provided seed. The seed
sequence then transforms the initial set of entropy into a sequence of
high-quality pseudorandom integers, which can be used to initialize multiple bit
generators deterministically. A key design goal of a seed sequence was to be
splittable in the sense that a seed sequence can be used to produce child
sequences that are distinct from their parents or other ancestors. This
capability allows a seed sequence to be used in large distributed applications
where the number of workers required is not known. The sequences generated from
the same initial entropy and the same splits are fully deterministic to ensure
reproducibility.

The three components are combined to construct a complete random number
generator.

\begin{lstlisting}
from numpy.random import Generator, PCG64, SeedSequence
seq = SeedSequence(10304245474441179933310169597522947680)
pcg = PCG64(seq)
gen = Generator(pcg)
\end{lstlisting}

\noindent This approach retains access to the seed sequence which can then be
used to spawn additional generators.

\begin{lstlisting}
children = seq.spawn(2)
gen_0 = Generator(PCG64(children[0]))
gen_1 = Generator(PCG64(children[1]))
\end{lstlisting}

\noindent While this approach retains complete flexibility, the method
\code{np.random.default\_rng} can be used to instantiate a \code{Generator} when
reproducibility is not needed.

The final goal of the new API is to improve extensibility. \code{RandomState} is
a monolithic object that obscures all of the underlying state and functions. The
component architecture is one part of the extensibility improvements. The
underlying functions (written in C) which transform the output of a bit
generator to other distributions are available for use in CFFI. This allows the
same code to be run in both NumPy and dependent that can consume CFFI, e.g.,
numba. Both the bit generators and the low-level functions can also be used in C
or Cython code.\footnote{As of 1.18.0, this scenario requires access to the
NumPy source. Alternative approaches that avoid this extra step are being
explored.} 

\subsection*{Testing on multiple architectures}

% started at as focused small matrices, class examples

At the time of writing the two fastest supercomputers in the
world, Summit and Sierra, both have IBM POWER9 architectures
\cite{top500nov2019}. In late 2018, Astra, the first ARM-based
supercomputer to enter the TOP500 list, went into production\cite{
astra-wiki}. Furthermore, over 100 billion ARM processors have been
produced as of 2017\cite{arm-architecture}, making it the most 
widely used instruction set architecture in the world.

Clearly there are motivations for a large scientific computing
software library to support POWER and ARM architectures. We've extended
our continuous integration (CI) testing to include \texttt{ppc64le}
(POWER8 on Travis CI) and ARMv8 (on Shippable service). We also test
with the s390x architecture (IBM Z CPUs on Travis CI) so that we
can probe the behavior of our library on a big-endian machine.
This satisfies one of the major components of
improved CI testing laid out in a version of our roadmap
\cite{numpy-roadmap}---specifically, ``CI for more exotic
platforms."

PEP 599\cite{PEP599} lays out a plan for new Python binary wheel
distribution support, \texttt{manylinux2014}, that adds
support for a number of architectures supported by the CentOS
Alternative Architecture Special Interest Group, including
ARMv8, ppc64le, as well as s390x. We are thus well-positioned
for a future where provision of binaries on these architectures
will be expected for a library at the base of the ecosystem.


\section*{Discussion}

% 1st grant done, technical cleanup finished except dtypes
% 2nd grant starting, community growth and new user focus

Most of the technical work for the Moore and Sloan grant is now complete.
The main remaining item is an overhaul of the data type system.

The meaning of each element within a NumPy array is described by its
datatype (dtype). NumPy arrays can hold all common numerical
datatypes, strings, datetimes, and generic Python objects, each of
which is identified by a dtype.  We are busy overhauling the datatype
system to make it behave consistently and to simplify the creation of
custom dtypes, both in C and in Python [XXX Cite the NEP here, instead
of \href{https://github.com/numpy/numpy/pull/14422}{draft of DTypes NEP}].
While numerical dtypes make the foundation for most users,
there are a multitude of use cases which cannot prosper due to current
limitations, such as physical units\cite{astropy,Goldbaum2018,pint},
% pyadolc may be a bit too small a project, so may want to remove/replace with an other example.
geometrical objects\cite{pygeos}, and automatic
differentiation\cite{pyadolc}.
%Projects addressing these exist, each struggles with current
%limitations.  % TODO: I may need citations, we could link github issues.

%% While it is currently possible to define custom dtypes within NumPy, many of the
%% use cases above cannot be solved using this system.
%% This is evidenced also by code within NumPy which would require changes
%% in many places to add such dtypes, even when adding them within NumPy itself.
%% The new implementation will be simpler and more cleanly implemented in
%% this regard.
%% Custom dtypes and the basic dtypes included within NumPy will be implemented
%% and behave identically as much as possible.
%% At the same time redesigning the API will make NumPy dtypes more flexible to grow
%% to future needs should they arise.

%% As a further step, to make dtypes more accessible and allow a faster adoption of
%% custom dtypes within the community, we plan to create a Python API to
%% define such custom dtypes.

Now that we have made good progress on addressing technical debt, the project
is ready to grow its community of contributors and to better address the needs
of an ever-growing number of new users.  This is the main focus of the
forthcoming CZI grant.

[... CZI grant ...]

[... welcome new users ...]

% technical focus
% improve various aspects of NumPy, including easier construction of custom data types (e.g., missing values, physical units, times & dates) and better integration with arrays customized for specialized domains (e.g., SciPy's sparse arrays, pandas' DataFrames, and xarray's labeled arrays). 
% technical work, and it increased the velocity of the projects by ~25-30%, and in addition it enabled integrating some larger changes, like the numpy.random redesign.


% \subsubsection*{Future Development}

% Dtypes

% outline plan for the CZI grant and an invitation to new contributors

%\subsection*{Website}

\section*{Data Availability}

All NumPy source code and most data generated for the current study are
available in the NumPy GitHub repository, \url{https://github.com/numpy}. Some
supporting code and data have also been stored in other public repositories
cited by this manuscript.

\bibliography{references}

\section*{Acknowledgments}


\section*{Author Contributions Statement}

%Must include all authors, identified by initials, for example: A.A.
%conceived the experiment(s),  A.A. and B.A. conducted the experiment(s), C.A.
%and D.A. analysed the results.  All authors reviewed the manuscript.

\section*{Competing Interests}

The authors declare no competing interests.

\section*{Consortium}
\subsection*{NumPy Contributors}

{\bfseries
Adam Apple$^{5}$
}
\newline
\hfill \break
$^{5}$International Centre for Ecological Exploration, California, USA


\end{document}

%%% Local Variables:
%%% mode: latex
%%% TeX-master: t
%%% End:
